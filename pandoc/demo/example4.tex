\documentclass[]{article}
\usepackage[T1]{fontenc}
\usepackage{lmodern}
\usepackage{amssymb,amsmath}
\usepackage{ifxetex,ifluatex}
\usepackage{fixltx2e} % provides \textsubscript
% use upquote if available, for straight quotes in verbatim environments
\IfFileExists{upquote.sty}{\usepackage{upquote}}{}
\ifnum 0\ifxetex 1\fi\ifluatex 1\fi=0 % if pdftex
  \usepackage[utf8]{inputenc}
\else % if luatex or xelatex
  \usepackage{fontspec}
  \ifxetex
    \usepackage{xltxtra,xunicode}
  \fi
  \defaultfontfeatures{Mapping=tex-text,Scale=MatchLowercase}
  \newcommand{\euro}{€}
\fi
% use microtype if available
\IfFileExists{microtype.sty}{\usepackage{microtype}}{}
\usepackage{fancyvrb}
\usepackage{longtable}
\ifxetex
  \usepackage[setpagesize=false, % page size defined by xetex
              unicode=false, % unicode breaks when used with xetex
              xetex]{hyperref}
\else
  \usepackage[unicode=true]{hyperref}
\fi
\hypersetup{breaklinks=true,
            bookmarks=true,
            pdfauthor={John MacFarlane},
            pdftitle={Pandoc User’s Guide},
            colorlinks=true,
            urlcolor=blue,
            linkcolor=magenta,
            pdfborder={0 0 0}}
\urlstyle{same}  % don't use monospace font for urls
\setlength{\parindent}{0pt}
\setlength{\parskip}{6pt plus 2pt minus 1pt}
\setlength{\emergencystretch}{3em}  % prevent overfull lines
\setcounter{secnumdepth}{0}
\VerbatimFootnotes % allows verbatim text in footnotes

\title{Pandoc User's Guide}
\author{John MacFarlane}
\date{January 19, 2013}

\begin{document}
\maketitle

\section{Synopsis}

pandoc {[}\emph{options}{]}{[}*input-file*{]}\ldots{}

\section{Description}

Pandoc is a \href{http://www.haskell.org/}{Haskell} library for
converting from one markup format to another, and a command-line tool
that uses this library. It can read
\href{http://daringfireball.net/projects/markdown/}{markdown} and
(subsets of) \href{http://redcloth.org/textile}{Textile},
\href{http://docutils.sourceforge.net/docs/ref/rst/introduction.html}{reStructuredText},
\href{http://www.w3.org/TR/html40/}{HTML},
\href{http://www.latex-project.org/}{LaTeX},
\href{http://www.mediawiki.org/wiki/Help:Formatting}{MediaWiki markup},
and \href{http://www.docbook.org/}{DocBook XML}; and it can write plain
text, \href{http://daringfireball.net/projects/markdown/}{markdown},
\href{http://docutils.sourceforge.net/docs/ref/rst/introduction.html}{reStructuredText},
\href{http://www.w3.org/TR/xhtml1/}{XHTML},
\href{http://www.w3.org/TR/html5/}{HTML 5},
\href{http://www.latex-project.org/}{LaTeX} (including
\href{http://www.tex.ac.uk/CTAN/macros/latex/contrib/beamer}{beamer}
slide shows), \href{http://www.pragma-ade.nl/}{ConTeXt},
\href{http://en.wikipedia.org/wiki/Rich_Text_Format}{RTF},
\href{http://www.docbook.org/}{DocBook XML},
\href{http://opendocument.xml.org/}{OpenDocument XML},
\href{http://en.wikipedia.org/wiki/OpenDocument}{ODT},
\href{http://www.microsoft.com/interop/openup/openxml/default.aspx}{Word
docx}, \href{http://www.gnu.org/software/texinfo/}{GNU Texinfo},
\href{http://www.mediawiki.org/wiki/Help:Formatting}{MediaWiki markup},
\href{http://www.idpf.org/}{EPUB} (v2 or v3),
\href{http://www.fictionbook.org/index.php/Eng:XML_Schema_Fictionbook_2.1}{FictionBook2},
\href{http://redcloth.org/textile}{Textile},
\href{http://developer.apple.com/DOCUMENTATION/Darwin/Reference/ManPages/man7/groff_man.7.html}{groff
man} pages, \href{http://orgmode.org}{Emacs Org-Mode},
\href{http://www.methods.co.nz/asciidoc/}{AsciiDoc}, and
\href{http://www.w3.org/Talks/Tools/Slidy/}{Slidy},
\href{http://goessner.net/articles/slideous/}{Slideous},
\href{http://paulrouget.com/dzslides/}{DZSlides}, or
\href{http://meyerweb.com/eric/tools/s5/}{S5} HTML slide shows. It can
also produce \href{http://www.adobe.com/pdf/}{PDF} output on systems
where LaTeX is installed.

Pandoc's enhanced version of markdown includes syntax for footnotes,
tables, flexible ordered lists, definition lists, fenced code blocks,
superscript, subscript, strikeout, title blocks, automatic tables of
contents, embedded LaTeX math, citations, and markdown inside HTML block
elements. (These enhancements, described below under
\hyperref[pandocs-markdown]{Pandoc's markdown}, can be disabled using
the \texttt{markdown\_strict} input or output format.)

In contrast to most existing tools for converting markdown to HTML,
which use regex substitutions, Pandoc has a modular design: it consists
of a set of readers, which parse text in a given format and produce a
native representation of the document, and a set of writers, which
convert this native representation into a target format. Thus, adding an
input or output format requires only adding a reader or writer.

\subsection{Using \texttt{pandoc}}

If no \emph{input-file} is specified, input is read from \emph{stdin}.
Otherwise, the \emph{input-files} are concatenated (with a blank line
between each) and used as input. Output goes to \emph{stdout} by default
(though output to \emph{stdout} is disabled for the \texttt{odt},
\texttt{docx}, \texttt{epub}, and \texttt{epub3} output formats). For
output to a file, use the \texttt{-o} option:

\begin{verbatim}
pandoc -o output.html input.txt
\end{verbatim}

Instead of a file, an absolute URI may be given. In this case pandoc
will fetch the content using HTTP:

\begin{verbatim}
pandoc -f html -t markdown http://www.fsf.org
\end{verbatim}

If multiple input files are given, \texttt{pandoc} will concatenate them
all (with blank lines between them) before parsing.

The format of the input and output can be specified explicitly using
command-line options. The input format can be specified using the
\texttt{-r/-{}-read} or \texttt{-f/-{}-from} options, the output format
using the \texttt{-w/-{}-write} or \texttt{-t/-{}-to} options. Thus, to
convert \texttt{hello.txt} from markdown to LaTeX, you could type:

\begin{verbatim}
pandoc -f markdown -t latex hello.txt
\end{verbatim}

To convert \texttt{hello.html} from html to markdown:

\begin{verbatim}
pandoc -f html -t markdown hello.html
\end{verbatim}

Supported output formats are listed below under the \texttt{-t/-{}-to}
option. Supported input formats are listed below under the
\texttt{-f/-{}-from} option. Note that the \texttt{rst},
\texttt{textile}, \texttt{latex}, and \texttt{html} readers are not
complete; there are some constructs that they do not parse.

If the input or output format is not specified explicitly,
\texttt{pandoc} will attempt to guess it from the extensions of the
input and output filenames. Thus, for example,

\begin{verbatim}
pandoc -o hello.tex hello.txt
\end{verbatim}

will convert \texttt{hello.txt} from markdown to LaTeX. If no output
file is specified (so that output goes to \emph{stdout}), or if the
output file's extension is unknown, the output format will default to
HTML. If no input file is specified (so that input comes from
\emph{stdin}), or if the input files' extensions are unknown, the input
format will be assumed to be markdown unless explicitly specified.

Pandoc uses the UTF-8 character encoding for both input and output. If
your local character encoding is not UTF-8, you should pipe input and
output through \texttt{iconv}:

\begin{verbatim}
iconv -t utf-8 input.txt | pandoc | iconv -f utf-8
\end{verbatim}

\subsection{Creating a PDF}

Earlier versions of pandoc came with a program, \texttt{markdown2pdf},
that used pandoc and pdflatex to produce a PDF. This is no longer
needed, since \texttt{pandoc} can now produce \texttt{pdf} output
itself. To produce a PDF, simply specify an output file with a
\texttt{.pdf} extension. Pandoc will create a latex file and use
pdflatex (or another engine, see \texttt{-{}-latex-engine}) to convert
it to PDF:

\begin{verbatim}
pandoc test.txt -o test.pdf
\end{verbatim}

Production of a PDF requires that a LaTeX engine be installed (see
\texttt{-{}-latex-engine}, below), and assumes that the following LaTeX
packages are available: \texttt{amssymb}, \texttt{amsmath},
\texttt{ifxetex}, \texttt{ifluatex}, \texttt{listings} (if the
\texttt{-{}-listings} option is used), \texttt{fancyvrb},
\texttt{longtable}, \texttt{url}, \texttt{graphicx}, \texttt{hyperref},
\texttt{ulem}, \texttt{babel} (if the \texttt{lang} variable is set),
\texttt{fontspec} (if \texttt{xelatex} or \texttt{lualatex} is used as
the LaTeX engine), \texttt{xltxtra} and \texttt{xunicode} (if
\texttt{xelatex} is used).

\subsection{\texttt{hsmarkdown}}

A user who wants a drop-in replacement for \texttt{Markdown.pl} may
create a symbolic link to the \texttt{pandoc} executable called
\texttt{hsmarkdown}. When invoked under the name \texttt{hsmarkdown},
\texttt{pandoc} will behave as if invoked with
\texttt{-f markdown\_strict -{}-email-obfuscation=references}, and all
command-line options will be treated as regular arguments. However, this
approach does not work under Cygwin, due to problems with its simulation
of symbolic links.

\section{Options}

\subsection{General options}

\begin{description}
\item[\texttt{-f} \emph{FORMAT}, \texttt{-r} \emph{FORMAT},
\texttt{-{}-from=}\emph{FORMAT}, \texttt{-{}-read=}\emph{FORMAT}]
Specify input format. \emph{FORMAT} can be \texttt{native} (native
Haskell), \texttt{json} (JSON version of native AST), \texttt{markdown}
(pandoc's extended markdown), \texttt{markdown\_strict} (original
unextended markdown), \texttt{markdown\_phpextra} (PHP Markdown Extra
extended markdown), \texttt{markdown\_github} (github extended
markdown), \texttt{textile} (Textile), \texttt{rst} (reStructuredText),
\texttt{html} (HTML), \texttt{docbook} (DocBook XML), \texttt{mediawiki}
(MediaWiki markup), or \texttt{latex} (LaTeX). If \texttt{+lhs} is
appended to \texttt{markdown}, \texttt{rst}, \texttt{latex}, the input
will be treated as literate Haskell source: see
\hyperref[literate-haskell-support]{Literate Haskell support}, below.
Markdown syntax extensions can be individually enabled or disabled by
appending \texttt{+EXTENSION} or \texttt{-EXTENSION} to the format name.
So, for example, \texttt{markdown\_strict+footnotes+definition\_lists}
is strict markdown with footnotes and definition lists enabled, and
\texttt{markdown-pipe\_tables+hard\_line\_breaks} is pandoc's markdown
without pipe tables and with hard line breaks. See
\hyperref[pandocs-markdown]{Pandoc's markdown}, below, for a list of
extensions and their names.
\item[\texttt{-t} \emph{FORMAT}, \texttt{-w} \emph{FORMAT},
\texttt{-{}-to=}\emph{FORMAT}, \texttt{-{}-write=}\emph{FORMAT}]
Specify output format. \emph{FORMAT} can be \texttt{native} (native
Haskell), \texttt{json} (JSON version of native AST), \texttt{plain}
(plain text), \texttt{markdown} (pandoc's extended markdown),
\texttt{markdown\_strict} (original unextended markdown),
\texttt{markdown\_phpextra} (PHP Markdown extra extended markdown),
\texttt{markdown\_github} (github extended markdown), \texttt{rst}
(reStructuredText), \texttt{html} (XHTML 1), \texttt{html5} (HTML 5),
\texttt{latex} (LaTeX), \texttt{beamer} (LaTeX beamer slide show),
\texttt{context} (ConTeXt), \texttt{man} (groff man), \texttt{mediawiki}
(MediaWiki markup), \texttt{textile} (Textile), \texttt{org} (Emacs
Org-Mode), \texttt{texinfo} (GNU Texinfo), \texttt{docbook} (DocBook
XML), \texttt{opendocument} (OpenDocument XML), \texttt{odt} (OpenOffice
text document), \texttt{docx} (Word docx), \texttt{epub} (EPUB book),
\texttt{epub3} (EPUB v3), \texttt{fb2} (FictionBook2 e-book),
\texttt{asciidoc} (AsciiDoc), \texttt{slidy} (Slidy HTML and javascript
slide show), \texttt{slideous} (Slideous HTML and javascript slide
show), \texttt{dzslides} (HTML5 + javascript slide show), \texttt{s5}
(S5 HTML and javascript slide show), or \texttt{rtf} (rich text format).
Note that \texttt{odt}, \texttt{epub}, and \texttt{epub3} output will
not be directed to \emph{stdout}; an output filename must be specified
using the \texttt{-o/-{}-output} option. If \texttt{+lhs} is appended to
\texttt{markdown}, \texttt{rst}, \texttt{latex}, \texttt{beamer},
\texttt{html}, or \texttt{html5}, the output will be rendered as
literate Haskell source: see
\hyperref[literate-haskell-support]{Literate Haskell support}, below.
Markdown syntax extensions can be individually enabled or disabled by
appending \texttt{+EXTENSION} or \texttt{-EXTENSION} to the format name,
as described above under \texttt{-f}.
\item[\texttt{-o} \emph{FILE}, \texttt{-{}-output=}\emph{FILE}]
Write output to \emph{FILE} instead of \emph{stdout}. If \emph{FILE} is
\texttt{-}, output will go to \emph{stdout}. (Exception: if the output
format is \texttt{odt}, \texttt{docx}, \texttt{epub}, or \texttt{epub3},
output to stdout is disabled.)
\item[\texttt{-{}-data-dir=}\emph{DIRECTORY}]
Specify the user data directory to search for pandoc data files. If this
option is not specified, the default user data directory will be used.
This is

\begin{verbatim}
$HOME/.pandoc
\end{verbatim}

in unix,

\begin{verbatim}
C:\Documents And Settings\USERNAME\Application Data\pandoc
\end{verbatim}

in Windows XP, and

\begin{verbatim}
C:\Users\USERNAME\AppData\Roaming\pandoc
\end{verbatim}

in Windows 7. (You can find the default user data directory on your
system by looking at the output of \texttt{pandoc -{}-version}.) A
\texttt{reference.odt}, \texttt{reference.docx}, \texttt{default.csl},
\texttt{epub.css}, \texttt{templates}, \texttt{slidy},
\texttt{slideous}, or \texttt{s5} directory placed in this directory
will override pandoc's normal defaults.
\item[\texttt{-v}, \texttt{-{}-version}]
Print version.
\item[\texttt{-h}, \texttt{-{}-help}]
Show usage message.
\end{description}

\subsection{Reader options}

\begin{description}
\item[\texttt{-R}, \texttt{-{}-parse-raw}]
Parse untranslatable HTML codes and LaTeX environments as raw HTML or
LaTeX, instead of ignoring them. Affects only HTML and LaTeX input. Raw
HTML can be printed in markdown, reStructuredText, HTML, Slidy,
Slideous, DZSlides, and S5 output; raw LaTeX can be printed in markdown,
reStructuredText, LaTeX, and ConTeXt output. The default is for the
readers to omit untranslatable HTML codes and LaTeX environments. (The
LaTeX reader does pass through untranslatable LaTeX \emph{commands},
even if \texttt{-R} is not specified.)
\item[\texttt{-S}, \texttt{-{}-smart}]
Produce typographically correct output, converting straight quotes to
curly quotes, \texttt{-{}-{}-} to em-dashes, \texttt{-{}-} to en-dashes,
and \texttt{...} to ellipses. Nonbreaking spaces are inserted after
certain abbreviations, such as ``Mr.'' (Note: This option is significant
only when the input format is \texttt{markdown},
\texttt{markdown\_strict}, or \texttt{textile}. It is selected
automatically when the input format is \texttt{textile} or the output
format is \texttt{latex} or \texttt{context}, unless
\texttt{-{}-no-tex-ligatures} is used.)
\item[\texttt{-{}-old-dashes}]
Selects the pandoc \textless{}= 1.8.2.1 behavior for parsing smart
dashes: \texttt{-} before a numeral is an en-dash, and \texttt{-{}-} is
an em-dash. This option is selected automatically for \texttt{textile}
input.
\item[\texttt{-{}-base-header-level=}\emph{NUMBER}]
Specify the base level for headers (defaults to 1).
\item[\texttt{-{}-indented-code-classes=}\emph{CLASSES}]
Specify classes to use for indented code blocks--for example,
\texttt{perl,numberLines} or \texttt{haskell}. Multiple classes may be
separated by spaces or commas.
\item[\texttt{-{}-default-image-extension=}\emph{EXTENSION}]
Specify a default extension to use when image paths/URLs have no
extension. This allows you to use the same source for formats that
require different kinds of images. Currently this option only affects
the markdown and LaTeX readers.
\item[\texttt{-{}-normalize}]
Normalize the document after reading: merge adjacent \texttt{Str} or
\texttt{Emph} elements, for example, and remove repeated
\texttt{Space}s.
\item[\texttt{-p}, \texttt{-{}-preserve-tabs}]
Preserve tabs instead of converting them to spaces (the default). Note
that this will only affect tabs in literal code spans and code blocks;
tabs in regular text will be treated as spaces.
\item[\texttt{-{}-tab-stop=}\emph{NUMBER}]
Specify the number of spaces per tab (default is 4).
\end{description}

\subsection{General writer options}

\begin{description}
\item[\texttt{-s}, \texttt{-{}-standalone}]
Produce output with an appropriate header and footer (e.g.~a standalone
HTML, LaTeX, or RTF file, not a fragment). This option is set
automatically for \texttt{pdf}, \texttt{epub}, \texttt{epub3},
\texttt{fb2}, \texttt{docx}, and \texttt{odt} output.
\item[\texttt{-{}-template=}\emph{FILE}]
Use \emph{FILE} as a custom template for the generated document. Implies
\texttt{-{}-standalone}. See \hyperref[templates]{Templates} below for a
description of template syntax. If no extension is specified, an
extension corresponding to the writer will be added, so that
\texttt{-{}-template=special} looks for \texttt{special.html} for HTML
output. If the template is not found, pandoc will search for it in the
user data directory (see \texttt{-{}-data-dir}). If this option is not
used, a default template appropriate for the output format will be used
(see \texttt{-D/-{}-print-default-template}).
\item[\texttt{-V} \emph{KEY{[}=VAL{]}},
\texttt{-{}-variable=}\emph{KEY{[}:VAL{]}}]
Set the template variable \emph{KEY} to the value \emph{VAL} when
rendering the document in standalone mode. This is generally only useful
when the \texttt{-{}-template} option is used to specify a custom
template, since pandoc automatically sets the variables used in the
default templates. If no \emph{VAL} is specified, the key will be given
the value \texttt{true}.
\item[\texttt{-D} \emph{FORMAT},
\texttt{-{}-print-default-template=}\emph{FORMAT}]
Print the default template for an output \emph{FORMAT}. (See \texttt{-t}
for a list of possible \emph{FORMAT}s.)
\item[\texttt{-{}-no-wrap}]
Disable text wrapping in output. By default, text is wrapped
appropriately for the output format.
\item[\texttt{-{}-columns}=\emph{NUMBER}]
Specify length of lines in characters (for text wrapping).
\item[\texttt{-{}-toc}, \texttt{-{}-table-of-contents}]
Include an automatically generated table of contents (or, in the case of
\texttt{latex}, \texttt{context}, and \texttt{rst}, an instruction to
create one) in the output document. This option has no effect on
\texttt{man}, \texttt{docbook}, \texttt{slidy}, \texttt{slideous}, or
\texttt{s5} output.
\item[\texttt{-{}-toc-depth=}\emph{NUMBER}]
Specify the number of section levels to include in the table of
contents. The default is 3 (which means that level 1, 2, and 3 headers
will be listed in the contents). Implies \texttt{-{}-toc}.
\item[\texttt{-{}-no-highlight}]
Disables syntax highlighting for code blocks and inlines, even when a
language attribute is given.
\item[\texttt{-{}-highlight-style}=\emph{STYLE}]
Specifies the coloring style to be used in highlighted source code.
Options are \texttt{pygments} (the default), \texttt{kate},
\texttt{monochrome}, \texttt{espresso}, \texttt{zenburn},
\texttt{haddock}, and \texttt{tango}.
\item[\texttt{-H} \emph{FILE},
\texttt{-{}-include-in-header=}\emph{FILE}]
Include contents of \emph{FILE}, verbatim, at the end of the header.
This can be used, for example, to include special CSS or javascript in
HTML documents. This option can be used repeatedly to include multiple
files in the header. They will be included in the order specified.
Implies \texttt{-{}-standalone}.
\item[\texttt{-B} \emph{FILE},
\texttt{-{}-include-before-body=}\emph{FILE}]
Include contents of \emph{FILE}, verbatim, at the beginning of the
document body (e.g.~after the \texttt{\textless{}body\textgreater{}} tag
in HTML, or the \texttt{\textbackslash{}begin\{document\}} command in
LaTeX). This can be used to include navigation bars or banners in HTML
documents. This option can be used repeatedly to include multiple files.
They will be included in the order specified. Implies
\texttt{-{}-standalone}.
\item[\texttt{-A} \emph{FILE},
\texttt{-{}-include-after-body=}\emph{FILE}]
Include contents of \emph{FILE}, verbatim, at the end of the document
body (before the \texttt{\textless{}/body\textgreater{}} tag in HTML, or
the \texttt{\textbackslash{}end\{document\}} command in LaTeX). This
option can be be used repeatedly to include multiple files. They will be
included in the order specified. Implies \texttt{-{}-standalone}.
\end{description}

\subsection{Options affecting specific writers}

\begin{description}
\item[\texttt{-{}-self-contained}]
Produce a standalone HTML file with no external dependencies, using
\texttt{data:} URIs to incorporate the contents of linked scripts,
stylesheets, images, and videos. The resulting file should be
``self-contained,'' in the sense that it needs no external files and no
net access to be displayed properly by a browser. This option works only
with HTML output formats, including \texttt{html}, \texttt{html5},
\texttt{html+lhs}, \texttt{html5+lhs}, \texttt{s5}, \texttt{slidy},
\texttt{slideous}, and \texttt{dzslides}. Scripts, images, and
stylesheets at absolute URLs will be downloaded; those at relative URLs
will be sought first relative to the working directory, then relative to
the user data directory (see \texttt{-{}-data-dir}), and finally
relative to pandoc's default data directory.
\item[\texttt{-{}-offline}]
Deprecated synonym for \texttt{-{}-self-contained}.
\item[\texttt{-5}, \texttt{-{}-html5}]
Produce HTML5 instead of HTML4. This option has no effect for writers
other than \texttt{html}. (\emph{Deprecated:} Use the \texttt{html5}
output format instead.)
\item[\texttt{-{}-html-q-tags}]
Use \texttt{\textless{}q\textgreater{}} tags for quotes in HTML.
\item[\texttt{-{}-ascii}]
Use only ascii characters in output. Currently supported only for HTML
output (which uses numerical entities instead of UTF-8 when this option
is selected).
\item[\texttt{-{}-reference-links}]
Use reference-style links, rather than inline links, in writing markdown
or reStructuredText. By default inline links are used.
\item[\texttt{-{}-atx-headers}]
Use ATX style headers in markdown output. The default is to use
setext-style headers for levels 1-2, and then ATX headers.
\item[\texttt{-{}-chapters}]
Treat top-level headers as chapters in LaTeX, ConTeXt, and DocBook
output. When the LaTeX template uses the report, book, or memoir class,
this option is implied. If \texttt{-{}-beamer} is used, top-level
headers will become \texttt{\textbackslash{}part\{..\}}.
\item[\texttt{-N}, \texttt{-{}-number-sections}]
Number section headings in LaTeX, ConTeXt, HTML, or EPUB output. By
default, sections are not numbered. Sections with class
\texttt{unnumbered} will never be numbered, even if
\texttt{-{}-number-sections} is specified.
\item[\texttt{-{}-number-offset}=\emph{NUMBER{[},NUMBER,\ldots{}{]}},]
Offset for section headings in HTML output (ignored in other output
formats). The first number is added to the section number for top-level
headers, the second for second-level headers, and so on. So, for
example, if you want the first top-level header in your document to be
numbered ``6'', specify \texttt{-{}-number-offset=5}. If your document
starts with a level-2 header which you want to be numbered ``1.5'',
specify \texttt{-{}-number-offset=1,4}. Offsets are 0 by default.
Implies \texttt{-{}-number-sections}.
\item[\texttt{-{}-no-tex-ligatures}]
Do not convert quotation marks, apostrophes, and dashes to the TeX
ligatures when writing LaTeX or ConTeXt. Instead, just use literal
unicode characters. This is needed for using advanced OpenType features
with XeLaTeX and LuaLaTeX. Note: normally \texttt{-{}-smart} is selected
automatically for LaTeX and ConTeXt output, but it must be specified
explicitly if \texttt{-{}-no-tex-ligatures} is selected. If you use
literal curly quotes, dashes, and ellipses in your source, then you may
want to use \texttt{-{}-no-tex-ligatures} without \texttt{-{}-smart}.
\item[\texttt{-{}-listings}]
Use listings package for LaTeX code blocks
\item[\texttt{-i}, \texttt{-{}-incremental}]
Make list items in slide shows display incrementally (one by one). The
default is for lists to be displayed all at once.
\item[\texttt{-{}-slide-level}=\emph{NUMBER}]
Specifies that headers with the specified level create slides (for
\texttt{beamer}, \texttt{s5}, \texttt{slidy}, \texttt{slideous},
\texttt{dzslides}). Headers above this level in the hierarchy are used
to divide the slide show into sections; headers below this level create
subheads within a slide. The default is to set the slide level based on
the contents of the document; see
\hyperref[structuring-the-slide-show]{Structuring the slide show},
below.
\item[\texttt{-{}-section-divs}]
Wrap sections in \texttt{\textless{}div\textgreater{}} tags (or
\texttt{\textless{}section\textgreater{}} tags in HTML5), and attach
identifiers to the enclosing \texttt{\textless{}div\textgreater{}} (or
\texttt{\textless{}section\textgreater{}}) rather than the header
itself. See
\hyperref[header-identifiers-in-html-latex-and-context]{Section
identifiers}, below.
\item[\texttt{-{}-email-obfuscation=}\emph{none\textbar{}javascript\textbar{}references}]
Specify a method for obfuscating \texttt{mailto:} links in HTML
documents. \emph{none} leaves \texttt{mailto:} links as they are.
\emph{javascript} obfuscates them using javascript. \emph{references}
obfuscates them by printing their letters as decimal or hexadecimal
character references.
\item[\texttt{-{}-id-prefix}=\emph{STRING}]
Specify a prefix to be added to all automatically generated identifiers
in HTML and DocBook output, and to footnote numbers in markdown output.
This is useful for preventing duplicate identifiers when generating
fragments to be included in other pages.
\item[\texttt{-T} \emph{STRING},
\texttt{-{}-title-prefix=}\emph{STRING}]
Specify \emph{STRING} as a prefix at the beginning of the title that
appears in the HTML header (but not in the title as it appears at the
beginning of the HTML body). Implies \texttt{-{}-standalone}.
\item[\texttt{-c} \emph{URL}, \texttt{-{}-css=}\emph{URL}]
Link to a CSS style sheet. This option can be be used repeatedly to
include multiple files. They will be included in the order specified.
\item[\texttt{-{}-reference-odt=}\emph{FILE}]
Use the specified file as a style reference in producing an ODT. For
best results, the reference ODT should be a modified version of an ODT
produced using pandoc. The contents of the reference ODT are ignored,
but its stylesheets are used in the new ODT. If no reference ODT is
specified on the command line, pandoc will look for a file
\texttt{reference.odt} in the user data directory (see
\texttt{-{}-data-dir}). If this is not found either, sensible defaults
will be used.
\item[\texttt{-{}-reference-docx=}\emph{FILE}]
Use the specified file as a style reference in producing a docx file.
For best results, the reference docx should be a modified version of a
docx file produced using pandoc. The contents of the reference docx are
ignored, but its stylesheets are used in the new docx. If no reference
docx is specified on the command line, pandoc will look for a file
\texttt{reference.docx} in the user data directory (see
\texttt{-{}-data-dir}). If this is not found either, sensible defaults
will be used. The following styles are used by pandoc: {[}paragraph{]}
Normal, Title, Authors, Date, Heading 1, Heading 2, Heading 3, Heading
4, Heading 5, Block Quote, Definition Term, Definition, Body Text, Table
Caption, Image Caption; {[}character{]} Default Paragraph Font, Body
Text Char, Verbatim Char, Footnote Ref, Link.
\item[\texttt{-{}-epub-stylesheet=}\emph{FILE}]
Use the specified CSS file to style the EPUB. If no stylesheet is
specified, pandoc will look for a file \texttt{epub.css} in the user
data directory (see \texttt{-{}-data-dir}). If it is not found there,
sensible defaults will be used.
\item[\texttt{-{}-epub-cover-image=}\emph{FILE}]
Use the specified image as the EPUB cover. It is recommended that the
image be less than 1000px in width and height.
\item[\texttt{-{}-epub-metadata=}\emph{FILE}]
Look in the specified XML file for metadata for the EPUB. The file
should contain a series of Dublin Core elements, as documented at
\url{http://dublincore.org/documents/dces/}. For example:

\begin{verbatim}
 <dc:rights>Creative Commons</dc:rights>
 <dc:language>es-AR</dc:language>
\end{verbatim}

By default, pandoc will include the following metadata elements:
\texttt{\textless{}dc:title\textgreater{}} (from the document title),
\texttt{\textless{}dc:creator\textgreater{}} (from the document
authors), \texttt{\textless{}dc:date\textgreater{}} (from the document
date, which should be in \href{http://www.w3.org/TR/NOTE-datetime}{ISO
8601 format}), \texttt{\textless{}dc:language\textgreater{}} (from the
\texttt{lang} variable, or, if is not set, the locale), and
\texttt{\textless{}dc:identifier id="BookId"\textgreater{}} (a randomly
generated UUID). Any of these may be overridden by elements in the
metadata file.
\item[\texttt{-{}-epub-embed-font=}\emph{FILE}]
Embed the specified font in the EPUB. This option can be repeated to
embed multiple fonts. To use embedded fonts, you will need to add
declarations like the following to your CSS (see
\texttt{-{}-epub-stylesheet}):

\begin{verbatim}
@font-face {
font-family: DejaVuSans;
font-style: normal;
font-weight: normal;
src:url("DejaVuSans-Regular.ttf");
}
@font-face {
font-family: DejaVuSans;
font-style: normal;
font-weight: bold;
src:url("DejaVuSans-Bold.ttf");
}
@font-face {
font-family: DejaVuSans;
font-style: italic;
font-weight: normal;
src:url("DejaVuSans-Oblique.ttf");
}
@font-face {
font-family: DejaVuSans;
font-style: italic;
font-weight: bold;
src:url("DejaVuSans-BoldOblique.ttf");
}
body { font-family: "DejaVuSans"; }
\end{verbatim}
\item[\texttt{-{}-epub-chapter-level=}\emph{NUMBER}]
Specify the header level at which to split the EPUB into separate
``chapter'' files. The default is to split into chapters at level 1
headers. This option only affects the internal composition of the EPUB,
not the way chapters and sections are displayed to users. Some readers
may be slow if the chapter files are too large, so for large documents
with few level 1 headers, one might want to use a chapter level of 2 or
3.
\item[\texttt{-{}-latex-engine=}\emph{pdflatex\textbar{}lualatex\textbar{}xelatex}]
Use the specified LaTeX engine when producing PDF output. The default is
\texttt{pdflatex}. If the engine is not in your PATH, the full path of
the engine may be specified here.
\end{description}

\subsection{Citation rendering}

\begin{description}
\item[\texttt{-{}-bibliography=}\emph{FILE}]
Specify bibliography database to be used in resolving citations. The
database type will be determined from the extension of \emph{FILE},
which may be \texttt{.mods} (MODS format), \texttt{.bib} (BibLaTeX
format, which will normally work for BibTeX files as well),
\texttt{.bibtex} (BibTeX format), \texttt{.ris} (RIS format),
\texttt{.enl} (EndNote format), \texttt{.xml} (EndNote XML format),
\texttt{.wos} (ISI format), \texttt{.medline} (MEDLINE format),
\texttt{.copac} (Copac format), or \texttt{.json} (citeproc JSON). If
you want to use multiple bibliographies, just use this option
repeatedly.
\item[\texttt{-{}-csl=}\emph{FILE}]
Specify \href{http://CitationStyles.org}{CSL} style to be used in
formatting citations and the bibliography. If \emph{FILE} is not found,
pandoc will look for it in

\begin{verbatim}
$HOME/.csl
\end{verbatim}

in unix,

\begin{verbatim}
C:\Documents And Settings\USERNAME\Application Data\csl
\end{verbatim}

in Windows XP, and

\begin{verbatim}
C:\Users\USERNAME\AppData\Roaming\csl
\end{verbatim}

in Windows 7. If the \texttt{-{}-csl} option is not specified, pandoc
will use a default style: either \texttt{default.csl} in the user data
directory (see \texttt{-{}-data-dir}), or, if that is not present, the
Chicago author-date style.
\item[\texttt{-{}-citation-abbreviations=}\emph{FILE}]
Specify a file containing abbreviations for journal titles and other
bibliographic fields (indicated by setting \texttt{form="short"} in the
CSL node for the field). The format is described at
\url{http://citationstylist.org/2011/10/19/abbreviations-for-zotero-test-release/}.
Here is a short example:

\begin{verbatim}
{ "default": {
    "container-title": {
            "Lloyd's Law Reports": "Lloyd's Rep",
            "Estates Gazette": "EG",
            "Scots Law Times": "SLT"
    }
  }
}
\end{verbatim}
\item[\texttt{-{}-natbib}]
Use natbib for citations in LaTeX output.
\item[\texttt{-{}-biblatex}]
Use biblatex for citations in LaTeX output.
\end{description}

\subsection{Math rendering in HTML}

\begin{description}
\item[\texttt{-m} {[}\emph{URL}{]},
\texttt{-{}-latexmathml}{[}=\emph{URL}{]}]
Use the \href{http://math.etsu.edu/LaTeXMathML/}{LaTeXMathML} script to
display embedded TeX math in HTML output. To insert a link to a local
copy of the \texttt{LaTeXMathML.js} script, provide a \emph{URL}. If no
\emph{URL} is provided, the contents of the script will be inserted
directly into the HTML header, preserving portability at the price of
efficiency. If you plan to use math on several pages, it is much better
to link to a copy of the script, so it can be cached.
\item[\texttt{-{}-mathml}{[}=\emph{URL}{]}]
Convert TeX math to MathML (in \texttt{docbook} as well as \texttt{html}
and \texttt{html5}). In standalone \texttt{html} output, a small
javascript (or a link to such a script if a \emph{URL} is supplied) will
be inserted that allows the MathML to be viewed on some browsers.
\item[\texttt{-{}-jsmath}{[}=\emph{URL}{]}]
Use \href{http://www.math.union.edu/~dpvc/jsmath/}{jsMath} to display
embedded TeX math in HTML output. The \emph{URL} should point to the
jsMath load script (e.g. \texttt{jsMath/easy/load.js}); if provided, it
will be linked to in the header of standalone HTML documents. If a
\emph{URL} is not provided, no link to the jsMath load script will be
inserted; it is then up to the author to provide such a link in the HTML
template.
\item[\texttt{-{}-mathjax}{[}=\emph{URL}{]}]
Use \href{http://www.mathjax.org/}{MathJax} to display embedded TeX math
in HTML output. The \emph{URL} should point to the \texttt{MathJax.js}
load script. If a \emph{URL} is not provided, a link to the MathJax CDN
will be inserted.
\item[\texttt{-{}-gladtex}]
Enclose TeX math in \texttt{\textless{}eq\textgreater{}} tags in HTML
output. These can then be processed by
\href{http://ans.hsh.no/home/mgg/gladtex/}{gladTeX} to produce links to
images of the typeset formulas.
\item[\texttt{-{}-mimetex}{[}=\emph{URL}{]}]
Render TeX math using the
\href{http://www.forkosh.com/mimetex.html}{mimeTeX} CGI script. If
\emph{URL} is not specified, it is assumed that the script is at
\texttt{/cgi-bin/mimetex.cgi}.
\item[\texttt{-{}-webtex}{[}=\emph{URL}{]}]
Render TeX formulas using an external script that converts TeX formulas
to images. The formula will be concatenated with the URL provided. If
\emph{URL} is not specified, the Google Chart API will be used.
\end{description}

\subsection{Options for wrapper scripts}

\begin{description}
\item[\texttt{-{}-dump-args}]
Print information about command-line arguments to \emph{stdout}, then
exit. This option is intended primarily for use in wrapper scripts. The
first line of output contains the name of the output file specified with
the \texttt{-o} option, or \texttt{-} (for \emph{stdout}) if no output
file was specified. The remaining lines contain the command-line
arguments, one per line, in the order they appear. These do not include
regular Pandoc options and their arguments, but do include any options
appearing after a \texttt{-{}-} separator at the end of the line.
\item[\texttt{-{}-ignore-args}]
Ignore command-line arguments (for use in wrapper scripts). Regular
Pandoc options are not ignored. Thus, for example,

\begin{verbatim}
pandoc --ignore-args -o foo.html -s foo.txt -- -e latin1
\end{verbatim}

is equivalent to

\begin{verbatim}
pandoc -o foo.html -s
\end{verbatim}
\end{description}

\hyperdef{}{templates}{\section{Templates}\label{templates}}

When the \texttt{-s/-{}-standalone} option is used, pandoc uses a
template to add header and footer material that is needed for a
self-standing document. To see the default template that is used, just
type

\begin{verbatim}
pandoc -D FORMAT
\end{verbatim}

where \texttt{FORMAT} is the name of the output format. A custom
template can be specified using the \texttt{-{}-template} option. You
can also override the system default templates for a given output format
\texttt{FORMAT} by putting a file \texttt{templates/default.FORMAT} in
the user data directory (see \texttt{-{}-data-dir}, above).
\emph{Exceptions:} For \texttt{odt} output, customize the
\texttt{default.opendocument} template. For \texttt{pdf} output,
customize the \texttt{default.latex} template.

Templates may contain \emph{variables}. Variable names are sequences of
alphanumerics, \texttt{-}, and \texttt{\_}, starting with a letter. A
variable name surrounded by \texttt{\$} signs will be replaced by its
value. For example, the string \texttt{\$title\$} in

\begin{verbatim}
<title>$title$</title>
\end{verbatim}

will be replaced by the document title.

To write a literal \texttt{\$} in a template, use \texttt{\$\$}.

Some variables are set automatically by pandoc. These vary somewhat
depending on the output format, but include:

\begin{description}
\itemsep1pt\parskip0pt\parsep0pt
\item[\texttt{header-includes}]
contents specified by \texttt{-H/-{}-include-in-header} (may have
multiple values)
\item[\texttt{toc}]
non-null value if \texttt{-{}-toc/-{}-table-of-contents} was specified
\item[\texttt{include-before}]
contents specified by \texttt{-B/-{}-include-before-body} (may have
multiple values)
\item[\texttt{include-after}]
contents specified by \texttt{-A/-{}-include-after-body} (may have
multiple values)
\item[\texttt{body}]
body of document
\item[\texttt{title}]
title of document, as specified in title block
\item[\texttt{author}]
author of document, as specified in title block (may have multiple
values)
\item[\texttt{date}]
date of document, as specified in title block
\item[\texttt{lang}]
language code for HTML or LaTeX documents
\item[\texttt{slidy-url}]
base URL for Slidy documents (defaults to
\texttt{http://www.w3.org/Talks/Tools/Slidy2})
\item[\texttt{slideous-url}]
base URL for Slideous documents (defaults to \texttt{default})
\item[\texttt{s5-url}]
base URL for S5 documents (defaults to \texttt{ui/default})
\item[\texttt{fontsize}]
font size (10pt, 11pt, 12pt) for LaTeX documents
\item[\texttt{documentclass}]
document class for LaTeX documents
\item[\texttt{geometry}]
options for LaTeX \texttt{geometry} class, e.g. \texttt{margin=1in}; may
be repeated for multiple options
\item[\texttt{mainfont}, \texttt{sansfont}, \texttt{monofont},
\texttt{mathfont}]
fonts for LaTeX documents (works only with xelatex and lualatex)
\item[\texttt{theme}]
theme for LaTeX beamer documents
\item[\texttt{colortheme}]
colortheme for LaTeX beamer documents
\item[\texttt{linkcolor}]
color for internal links in LaTeX documents (\texttt{red},
\texttt{green}, \texttt{magenta}, \texttt{cyan}, \texttt{blue},
\texttt{black})
\item[\texttt{urlcolor}]
color for external links in LaTeX documents
\item[\texttt{links-as-notes}]
causes links to be printed as footnotes in LaTeX documents
\end{description}

Variables may be set at the command line using the
\texttt{-V/-{}-variable} option. This allows users to include custom
variables in their templates.

Templates may contain conditionals. The syntax is as follows:

\begin{verbatim}
$if(variable)$
X
$else$
Y
$endif$
\end{verbatim}

This will include \texttt{X} in the template if \texttt{variable} has a
non-null value; otherwise it will include \texttt{Y}. \texttt{X} and
\texttt{Y} are placeholders for any valid template text, and may include
interpolated variables or other conditionals. The \texttt{\$else\$}
section may be omitted.

When variables can have multiple values (for example, \texttt{author} in
a multi-author document), you can use the \texttt{\$for\$} keyword:

\begin{verbatim}
$for(author)$
<meta name="author" content="$author$" />
$endfor$
\end{verbatim}

You can optionally specify a separator to be used between consecutive
items:

\begin{verbatim}
$for(author)$$author$$sep$, $endfor$
\end{verbatim}

If you use custom templates, you may need to revise them as pandoc
changes. We recommend tracking the changes in the default templates, and
modifying your custom templates accordingly. An easy way to do this is
to fork the pandoc-templates repository
(\url{http://github.com/jgm/pandoc-templates}) and merge in changes
after each pandoc release.

\hyperdef{}{pandocs-markdown}{\section{Pandoc's
markdown}\label{pandocs-markdown}}

Pandoc understands an extended and slightly revised version of John
Gruber's \href{http://daringfireball.net/projects/markdown/}{markdown}
syntax. This document explains the syntax, noting differences from
standard markdown. Except where noted, these differences can be
suppressed by using the \texttt{markdown\_strict} format instead of
\texttt{markdown}. An extensions can be enabled by adding
\texttt{+EXTENSION} to the format name and disabled by adding
\texttt{-EXTENSION}. For example, \texttt{markdown\_strict+footnotes} is
strict markdown with footnotes enabled, while
\texttt{markdown-footnotes-pipe\_tables} is pandoc's markdown without
footnotes or pipe tables.

\subsection{Philosophy}

Markdown is designed to be easy to write, and, even more importantly,
easy to read:

\begin{quote}
A Markdown-formatted document should be publishable as-is, as plain
text, without looking like it's been marked up with tags or formatting
instructions. --
\href{http://daringfireball.net/projects/markdown/syntax\#philosophy}{John
Gruber}
\end{quote}

This principle has guided pandoc's decisions in finding syntax for
tables, footnotes, and other extensions.

There is, however, one respect in which pandoc's aims are different from
the original aims of markdown. Whereas markdown was originally designed
with HTML generation in mind, pandoc is designed for multiple output
formats. Thus, while pandoc allows the embedding of raw HTML, it
discourages it, and provides other, non-HTMLish ways of representing
important document elements like definition lists, tables, mathematics,
and footnotes.

\subsection{Paragraphs}

A paragraph is one or more lines of text followed by one or more blank
line. Newlines are treated as spaces, so you can reflow your paragraphs
as you like. If you need a hard line break, put two or more spaces at
the end of a line.

\textbf{Extension: \texttt{escaped\_line\_breaks}}

A backslash followed by a newline is also a hard line break.

\subsection{Headers}

There are two kinds of headers, Setext and atx.

\subsubsection{Setext-style headers}

A setext-style header is a line of text ``underlined'' with a row of
\texttt{=} signs (for a level one header) of \texttt{-} signs (for a
level two header):

\begin{verbatim}
A level-one header
==================

A level-two header
------------------
\end{verbatim}

The header text can contain inline formatting, such as emphasis (see
\hyperref[inline-formatting]{Inline formatting}, below).

\subsubsection{Atx-style headers}

An Atx-style header consists of one to six \texttt{\#} signs and a line
of text, optionally followed by any number of \texttt{\#} signs. The
number of \texttt{\#} signs at the beginning of the line is the header
level:

\begin{verbatim}
## A level-two header

### A level-three header ###
\end{verbatim}

As with setext-style headers, the header text can contain formatting:

\begin{verbatim}
# A level-one header with a [link](/url) and *emphasis*
\end{verbatim}

\textbf{Extension: \texttt{blank\_before\_header}}

Standard markdown syntax does not require a blank line before a header.
Pandoc does require this (except, of course, at the beginning of the
document). The reason for the requirement is that it is all too easy for
a \texttt{\#} to end up at the beginning of a line by accident (perhaps
through line wrapping). Consider, for example:

\begin{verbatim}
I like several of their flavors of ice cream:
#22, for example, and #5.
\end{verbatim}

\hyperdef{}{header-identifiers-in-html-latex-and-context}{\subsubsection{Header
identifiers in HTML, LaTeX, and
ConTeXt}\label{header-identifiers-in-html-latex-and-context}}

\textbf{Extension: \texttt{header\_attributes}}

Headers can be assigned attributes using this syntax at the end of the
line containing the header text:

\begin{verbatim}
{#identifier .class .class key=value key=value}
\end{verbatim}

Although this syntax allows assignment of classes and key/value
attributes, only identifiers currently have any affect in the writers
(and only in some writers: HTML, LaTeX, ConTeXt, Textile, AsciiDoc).
Thus, for example, the following headers will all be assigned the
identifier \texttt{foo}:

\begin{verbatim}
# My header {#foo}

## My header ##    {#foo}

My other header   {#foo}
---------------
\end{verbatim}

(This syntax is compatible with
\href{http://www.michelf.com/projects/php-markdown/extra/}{PHP Markdown
Extra}.)

Headers with the class \texttt{unnumbered} will not be numbered, even if
\texttt{-{}-number-sections} is specified. A single hyphen (\texttt{-})
in an attribute context is equivalent to \texttt{.unnumbered}, and
preferable in non-English documents. So,

\begin{verbatim}
# My header {-}
\end{verbatim}

is just the same as

\begin{verbatim}
# My header {.unnumbered}
\end{verbatim}

\textbf{Extension: \texttt{auto\_identifiers}}

A header without an explicitly specified identifier will be
automatically assigned a unique identifier based on the header text. To
derive the identifier from the header text,

\begin{itemize}
\itemsep1pt\parskip0pt\parsep0pt
\item
  Remove all formatting, links, etc.
\item
  Remove all punctuation, except underscores, hyphens, and periods.
\item
  Replace all spaces and newlines with hyphens.
\item
  Convert all alphabetic characters to lowercase.
\item
  Remove everything up to the first letter (identifiers may not begin
  with a number or punctuation mark).
\item
  If nothing is left after this, use the identifier \texttt{section}.
\end{itemize}

Thus, for example,

\begin{longtable}[c]{@{}ll@{}}
\hline\noalign{\medskip}
Header & Identifier
\\\noalign{\medskip}
\hline\noalign{\medskip}
Header identifiers in HTML & \texttt{header-identifiers-in-html}
\\\noalign{\medskip}
\emph{Dogs}?--in \emph{my} house? & \texttt{dogs-{}-in-my-house}
\\\noalign{\medskip}
\href{http://www.w3.org/TR/html40/}{HTML},
\href{http://meyerweb.com/eric/tools/s5/}{S5}, or
\href{http://en.wikipedia.org/wiki/Rich_Text_Format}{RTF}? &
\texttt{html-s5-or-rtf}
\\\noalign{\medskip}
3. Applications & \texttt{applications}
\\\noalign{\medskip}
33 & \texttt{section}
\\\noalign{\medskip}
\hline
\end{longtable}

These rules should, in most cases, allow one to determine the identifier
from the header text. The exception is when several headers have the
same text; in this case, the first will get an identifier as described
above; the second will get the same identifier with \texttt{-1}
appended; the third with \texttt{-2}; and so on.

These identifiers are used to provide link targets in the table of
contents generated by the
\texttt{-{}-toc\textbar{}-{}-table-of-contents} option. They also make
it easy to provide links from one section of a document to another. A
link to this section, for example, might look like this:

\begin{verbatim}
See the section on
[header identifiers](#header-identifiers-in-html-latex-and-context).
\end{verbatim}

Note, however, that this method of providing links to sections works
only in HTML, LaTeX, and ConTeXt formats.

If the \texttt{-{}-section-divs} option is specified, then each section
will be wrapped in a \texttt{div} (or a \texttt{section}, if
\texttt{-{}-html5} was specified), and the identifier will be attached
to the enclosing \texttt{\textless{}div\textgreater{}} (or
\texttt{\textless{}section\textgreater{}}) tag rather than the header
itself. This allows entire sections to be manipulated using javascript
or treated differently in CSS.

\textbf{Extension: \texttt{implicit\_header\_references}}

Pandoc behaves as if reference links have been defined for each header.
So, instead of

\begin{verbatim}
[header identifiers](#header-identifiers-in-html)
\end{verbatim}

you can simply write

\begin{verbatim}
[header identifiers]
\end{verbatim}

or

\begin{verbatim}
[header identifiers][]
\end{verbatim}

or

\begin{verbatim}
[the section on header identifiers][header identifiers]
\end{verbatim}

If there are multiple headers with identical text, the corresponding
reference will link to the first one only, and you will need to use
explicit links to link to the others, as described above.

Unlike regular reference links, these references are case-sensitive.

Note: if you have defined an explicit identifier for a header, then
implicit references to it will not work.

\subsection{Block quotations}

Markdown uses email conventions for quoting blocks of text. A block
quotation is one or more paragraphs or other block elements (such as
lists or headers), with each line preceded by a \texttt{\textgreater{}}
character and a space. (The \texttt{\textgreater{}} need not start at
the left margin, but it should not be indented more than three spaces.)

\begin{verbatim}
> This is a block quote. This
> paragraph has two lines.
>
> 1. This is a list inside a block quote.
> 2. Second item.
\end{verbatim}

A ``lazy'' form, which requires the \texttt{\textgreater{}} character
only on the first line of each block, is also allowed:

\begin{verbatim}
> This is a block quote. This
paragraph has two lines.

> 1. This is a list inside a block quote.
2. Second item.
\end{verbatim}

Among the block elements that can be contained in a block quote are
other block quotes. That is, block quotes can be nested:

\begin{verbatim}
> This is a block quote.
>
> > A block quote within a block quote.
\end{verbatim}

\textbf{Extension: \texttt{blank\_before\_blockquote}}

Standard markdown syntax does not require a blank line before a block
quote. Pandoc does require this (except, of course, at the beginning of
the document). The reason for the requirement is that it is all too easy
for a \texttt{\textgreater{}} to end up at the beginning of a line by
accident (perhaps through line wrapping). So, unless the
\texttt{markdown\_strict} format is used, the following does not produce
a nested block quote in pandoc:

\begin{verbatim}
> This is a block quote.
>> Nested.
\end{verbatim}

\subsection{Verbatim (code) blocks}

\subsubsection{Indented code blocks}

A block of text indented four spaces (or one tab) is treated as verbatim
text: that is, special characters do not trigger special formatting, and
all spaces and line breaks are preserved. For example,

\begin{verbatim}
    if (a > 3) {
      moveShip(5 * gravity, DOWN);
    }
\end{verbatim}

The initial (four space or one tab) indentation is not considered part
of the verbatim text, and is removed in the output.

Note: blank lines in the verbatim text need not begin with four spaces.

\hyperdef{}{fenced-code-blocks}{\subsubsection{Fenced code
blocks}\label{fenced-code-blocks}}

\textbf{Extension: \texttt{fenced\_code\_blocks}}

In addition to standard indented code blocks, Pandoc supports
\emph{fenced} code blocks. These begin with a row of three or more
tildes (\texttt{\textasciitilde{}}) or backticks (\texttt{`}) and end
with a row of tildes or backticks that must be at least as long as the
starting row. Everything between these lines is treated as code. No
indentation is necessary:

\begin{verbatim}
~~~~~~~
if (a > 3) {
  moveShip(5 * gravity, DOWN);
}
~~~~~~~
\end{verbatim}

Like regular code blocks, fenced code blocks must be separated from
surrounding text by blank lines.

If the code itself contains a row of tildes or backticks, just use a
longer row of tildes or backticks at the start and end:

\begin{verbatim}
~~~~~~~~~~~~~~~~
~~~~~~~~~~
code including tildes
~~~~~~~~~~
~~~~~~~~~~~~~~~~
\end{verbatim}

Optionally, you may attach attributes to the code block using this
syntax:

\begin{verbatim}
~~~~ {#mycode .haskell .numberLines startFrom="100"}
qsort []     = []
qsort (x:xs) = qsort (filter (< x) xs) ++ [x] ++
               qsort (filter (>= x) xs)
~~~~~~~~~~~~~~~~~~~~~~~~~~~~~~~~~~~~~~~~~~~~~~~~~
\end{verbatim}

Here \texttt{mycode} is an identifier, \texttt{haskell} and
\texttt{numberLines} are classes, and \texttt{startFrom} is an attribute
with value \texttt{100}. Some output formats can use this information to
do syntax highlighting. Currently, the only output formats that uses
this information are HTML and LaTeX. If highlighting is supported for
your output format and language, then the code block above will appear
highlighted, with numbered lines. (To see which languages are supported,
do \texttt{pandoc -{}-version}.) Otherwise, the code block above will
appear as follows:

\begin{verbatim}
<pre id="mycode" class="haskell numberLines" startFrom="100">
  <code>
  ...
  </code>
</pre>
\end{verbatim}

A shortcut form can also be used for specifying the language of the code
block:

\begin{verbatim}
```haskell
qsort [] = []
```
\end{verbatim}

This is equivalent to:

\begin{verbatim}
``` {.haskell}
qsort [] = []
```
\end{verbatim}

To prevent all highlighting, use the \texttt{-{}-no-highlight} flag. To
set the highlighting style, use \texttt{-{}-highlight-style}.

\subsection{Line blocks}

\textbf{Extension: \texttt{line\_blocks}}

A line block is a sequence of lines beginning with a vertical bar
(\texttt{\textbar{}}) followed by a space. The division into lines will
be preserved in the output, as will any leading spaces; otherwise, the
lines will be formatted as markdown. This is useful for verse and
addresses:

\begin{verbatim}
| The limerick packs laughs anatomical
| In space that is quite economical.
|    But the good ones I've seen
|    So seldom are clean
| And the clean ones so seldom are comical

| 200 Main St.
| Berkeley, CA 94718
\end{verbatim}

The lines can be hard-wrapped if needed, but the continuation line must
begin with a space.

\begin{verbatim}
| The Right Honorable Most Venerable and Righteous Samuel L.
  Constable, Jr.
| 200 Main St.
| Berkeley, CA 94718
\end{verbatim}

This syntax is borrowed from
\href{http://docutils.sourceforge.net/docs/ref/rst/introduction.html}{reStructuredText}.

\subsection{Lists}

\subsubsection{Bullet lists}

A bullet list is a list of bulleted list items. A bulleted list item
begins with a bullet (\texttt{*}, \texttt{+}, or \texttt{-}). Here is a
simple example:

\begin{verbatim}
* one
* two
* three
\end{verbatim}

This will produce a ``compact'' list. If you want a ``loose'' list, in
which each item is formatted as a paragraph, put spaces between the
items:

\begin{verbatim}
* one

* two

* three
\end{verbatim}

The bullets need not be flush with the left margin; they may be indented
one, two, or three spaces. The bullet must be followed by whitespace.

List items look best if subsequent lines are flush with the first line
(after the bullet):

\begin{verbatim}
* here is my first
  list item.
* and my second.
\end{verbatim}

But markdown also allows a ``lazy'' format:

\begin{verbatim}
* here is my first
list item.
* and my second.
\end{verbatim}

\subsubsection{The four-space rule}

A list item may contain multiple paragraphs and other block-level
content. However, subsequent paragraphs must be preceded by a blank line
and indented four spaces or a tab. The list will look better if the
first paragraph is aligned with the rest:

\begin{verbatim}
  * First paragraph.

    Continued.

  * Second paragraph. With a code block, which must be indented
    eight spaces:

        { code }
\end{verbatim}

List items may include other lists. In this case the preceding blank
line is optional. The nested list must be indented four spaces or one
tab:

\begin{verbatim}
* fruits
    + apples
        - macintosh
        - red delicious
    + pears
    + peaches
* vegetables
    + brocolli
    + chard
\end{verbatim}

As noted above, markdown allows you to write list items ``lazily,''
instead of indenting continuation lines. However, if there are multiple
paragraphs or other blocks in a list item, the first line of each must
be indented.

\begin{verbatim}
+ A lazy, lazy, list
item.

+ Another one; this looks
bad but is legal.

    Second paragraph of second
list item.
\end{verbatim}

\textbf{Note:} Although the four-space rule for continuation paragraphs
comes from the official
\href{http://daringfireball.net/projects/markdown/syntax\#list}{markdown
syntax guide}, the reference implementation, \texttt{Markdown.pl}, does
not follow it. So pandoc will give different results than
\texttt{Markdown.pl} when authors have indented continuation paragraphs
fewer than four spaces.

The
\href{http://daringfireball.net/projects/markdown/syntax\#list}{markdown
syntax guide} is not explicit whether the four-space rule applies to
\emph{all} block-level content in a list item; it only mentions
paragraphs and code blocks. But it implies that the rule applies to all
block-level content (including nested lists), and pandoc interprets it
that way.

\subsubsection{Ordered lists}

Ordered lists work just like bulleted lists, except that the items begin
with enumerators rather than bullets.

In standard markdown, enumerators are decimal numbers followed by a
period and a space. The numbers themselves are ignored, so there is no
difference between this list:

\begin{verbatim}
1.  one
2.  two
3.  three
\end{verbatim}

and this one:

\begin{verbatim}
5.  one
7.  two
1.  three
\end{verbatim}

\textbf{Extension: \texttt{fancy\_lists}}

Unlike standard markdown, Pandoc allows ordered list items to be marked
with uppercase and lowercase letters and roman numerals, in addition to
arabic numerals. List markers may be enclosed in parentheses or followed
by a single right-parentheses or period. They must be separated from the
text that follows by at least one space, and, if the list marker is a
capital letter with a period, by at least two spaces.\footnote{The point
  of this rule is to ensure that normal paragraphs starting with
  people's initials, like

\begin{Verbatim}
B. Russell was an English philosopher.
\end{Verbatim}

  do not get treated as list items.

  This rule will not prevent

\begin{Verbatim}
(C) 2007 Joe Smith
\end{Verbatim}

  from being interpreted as a list item. In this case, a backslash
  escape can be used:

\begin{Verbatim}
(C\) 2007 Joe Smith
\end{Verbatim}
}

\textbf{Extension: \texttt{startnum}}

Pandoc also pays attention to the type of list marker used, and to the
starting number, and both of these are preserved where possible in the
output format. Thus, the following yields a list with numbers followed
by a single parenthesis, starting with 9, and a sublist with lowercase
roman numerals:

\begin{verbatim}
 9)  Ninth
10)  Tenth
11)  Eleventh
       i. subone
      ii. subtwo
     iii. subthree
\end{verbatim}

Pandoc will start a new list each time a different type of list marker
is used. So, the following will create three lists:

\begin{verbatim}
(2) Two
(5) Three
1.  Four
*   Five
\end{verbatim}

If default list markers are desired, use \texttt{\#.}:

\begin{verbatim}
#.  one
#.  two
#.  three
\end{verbatim}

\subsubsection{Definition lists}

\textbf{Extension: \texttt{definition\_lists}}

Pandoc supports definition lists, using a syntax inspired by
\href{http://www.michelf.com/projects/php-markdown/extra/}{PHP Markdown
Extra} and
\href{http://docutils.sourceforge.net/docs/ref/rst/introduction.html}{reStructuredText}:\footnote{I
  have also been influenced by the suggestions of
  \href{http://www.justatheory.com/computers/markup/modest-markdown-proposal.html}{David
  Wheeler}.}

\begin{verbatim}
Term 1

:   Definition 1

Term 2 with *inline markup*

:   Definition 2

        { some code, part of Definition 2 }

    Third paragraph of definition 2.
\end{verbatim}

Each term must fit on one line, which may optionally be followed by a
blank line, and must be followed by one or more definitions. A
definition begins with a colon or tilde, which may be indented one or
two spaces. The body of the definition (including the first line, aside
from the colon or tilde) should be indented four spaces. A term may have
multiple definitions, and each definition may consist of one or more
block elements (paragraph, code block, list, etc.), each indented four
spaces or one tab stop.

If you leave space after the definition (as in the example above), the
blocks of the definitions will be considered paragraphs. In some output
formats, this will mean greater spacing between term/definition pairs.
For a compact definition list, do not leave space between the definition
and the next term:

\begin{verbatim}
Term 1
  ~ Definition 1
Term 2
  ~ Definition 2a
  ~ Definition 2b
\end{verbatim}

\subsubsection{Numbered example lists}

\textbf{Extension: \texttt{example\_lists}}

The special list marker \texttt{@} can be used for sequentially numbered
examples. The first list item with a \texttt{@} marker will be numbered
`1', the next `2', and so on, throughout the document. The numbered
examples need not occur in a single list; each new list using \texttt{@}
will take up where the last stopped. So, for example:

\begin{verbatim}
(@)  My first example will be numbered (1).
(@)  My second example will be numbered (2).

Explanation of examples.

(@)  My third example will be numbered (3).
\end{verbatim}

Numbered examples can be labeled and referred to elsewhere in the
document:

\begin{verbatim}
(@good)  This is a good example.

As (@good) illustrates, ...
\end{verbatim}

The label can be any string of alphanumeric characters, underscores, or
hyphens.

\subsubsection{Compact and loose lists}

Pandoc behaves differently from \texttt{Markdown.pl} on some ``edge
cases'' involving lists. Consider this source:

\begin{verbatim}
+   First
+   Second:
    -   Fee
    -   Fie
    -   Foe

+   Third
\end{verbatim}

Pandoc transforms this into a ``compact list'' (with no
\texttt{\textless{}p\textgreater{}} tags around ``First'', ``Second'',
or ``Third''), while markdown puts \texttt{\textless{}p\textgreater{}}
tags around ``Second'' and ``Third'' (but not ``First''), because of the
blank space around ``Third''. Pandoc follows a simple rule: if the text
is followed by a blank line, it is treated as a paragraph. Since
``Second'' is followed by a list, and not a blank line, it isn't treated
as a paragraph. The fact that the list is followed by a blank line is
irrelevant. (Note: Pandoc works this way even when the
\texttt{markdown\_strict} format is specified. This behavior is
consistent with the official markdown syntax description, even though it
is different from that of \texttt{Markdown.pl}.)

\subsubsection{Ending a list}

What if you want to put an indented code block after a list?

\begin{verbatim}
-   item one
-   item two

    { my code block }
\end{verbatim}

Trouble! Here pandoc (like other markdown implementations) will treat
\texttt{\{ my code block \}} as the second paragraph of item two, and
not as a code block.

To ``cut off'' the list after item two, you can insert some non-indented
content, like an HTML comment, which won't produce visible output in any
format:

\begin{verbatim}
-   item one
-   item two

<!-- end of list -->

    { my code block }
\end{verbatim}

You can use the same trick if you want two consecutive lists instead of
one big list:

\begin{verbatim}
1.  one
2.  two
3.  three

<!-- -->

1.  uno
2.  dos
3.  tres
\end{verbatim}

\subsection{Horizontal rules}

A line containing a row of three or more \texttt{*}, \texttt{-}, or
\texttt{\_} characters (optionally separated by spaces) produces a
horizontal rule:

\begin{verbatim}
*  *  *  *

---------------
\end{verbatim}

\subsection{Tables}

Four kinds of tables may be used. The first three kinds presuppose the
use of a fixed-width font, such as Courier. The fourth kind can be used
with proportionally spaced fonts, as it does not require lining up
columns.

\subsubsection{Simple tables}

\textbf{Extension: \texttt{simple\_tables}, \texttt{table\_captions}}

Simple tables look like this:

\begin{verbatim}
  Right     Left     Center     Default
-------     ------ ----------   -------
     12     12        12            12
    123     123       123          123
      1     1          1             1

Table:  Demonstration of simple table syntax.
\end{verbatim}

The headers and table rows must each fit on one line. Column alignments
are determined by the position of the header text relative to the dashed
line below it:\footnote{This scheme is due to Michel Fortin, who
  proposed it on the
  \href{http://six.pairlist.net/pipermail/markdown-discuss/2005-March/001097.html}{Markdown
  discussion list}.}

\begin{itemize}
\itemsep1pt\parskip0pt\parsep0pt
\item
  If the dashed line is flush with the header text on the right side but
  extends beyond it on the left, the column is right-aligned.
\item
  If the dashed line is flush with the header text on the left side but
  extends beyond it on the right, the column is left-aligned.
\item
  If the dashed line extends beyond the header text on both sides, the
  column is centered.
\item
  If the dashed line is flush with the header text on both sides, the
  default alignment is used (in most cases, this will be left).
\end{itemize}

The table must end with a blank line, or a line of dashes followed by a
blank line. A caption may optionally be provided (as illustrated in the
example above). A caption is a paragraph beginning with the string
\texttt{Table:} (or just \texttt{:}), which will be stripped off. It may
appear either before or after the table.

The column headers may be omitted, provided a dashed line is used to end
the table. For example:

\begin{verbatim}
-------     ------ ----------   -------
     12     12        12             12
    123     123       123           123
      1     1          1              1
-------     ------ ----------   -------
\end{verbatim}

When headers are omitted, column alignments are determined on the basis
of the first line of the table body. So, in the tables above, the
columns would be right, left, center, and right aligned, respectively.

\subsubsection{Multiline tables}

\textbf{Extension: \texttt{multiline\_tables}, \texttt{table\_captions}}

Multiline tables allow headers and table rows to span multiple lines of
text (but cells that span multiple columns or rows of the table are not
supported). Here is an example:

\begin{verbatim}
-------------------------------------------------------------
 Centered   Default           Right Left
  Header    Aligned         Aligned Aligned
----------- ------- --------------- -------------------------
   First    row                12.0 Example of a row that
                                    spans multiple lines.

  Second    row                 5.0 Here's another one. Note
                                    the blank line between
                                    rows.
-------------------------------------------------------------

Table: Here's the caption. It, too, may span
multiple lines.
\end{verbatim}

These work like simple tables, but with the following differences:

\begin{itemize}
\itemsep1pt\parskip0pt\parsep0pt
\item
  They must begin with a row of dashes, before the header text (unless
  the headers are omitted).
\item
  They must end with a row of dashes, then a blank line.
\item
  The rows must be separated by blank lines.
\end{itemize}

In multiline tables, the table parser pays attention to the widths of
the columns, and the writers try to reproduce these relative widths in
the output. So, if you find that one of the columns is too narrow in the
output, try widening it in the markdown source.

Headers may be omitted in multiline tables as well as simple tables:

\begin{verbatim}
----------- ------- --------------- -------------------------
   First    row                12.0 Example of a row that
                                    spans multiple lines.

  Second    row                 5.0 Here's another one. Note
                                    the blank line between
                                    rows.
----------- ------- --------------- -------------------------

: Here's a multiline table without headers.
\end{verbatim}

It is possible for a multiline table to have just one row, but the row
should be followed by a blank line (and then the row of dashes that ends
the table), or the table may be interpreted as a simple table.

\subsubsection{Grid tables}

\textbf{Extension: \texttt{grid\_tables}, \texttt{table\_captions}}

Grid tables look like this:

\begin{verbatim}
: Sample grid table.

+---------------+---------------+--------------------+
| Fruit         | Price         | Advantages         |
+===============+===============+====================+
| Bananas       | $1.34         | - built-in wrapper |
|               |               | - bright color     |
+---------------+---------------+--------------------+
| Oranges       | $2.10         | - cures scurvy     |
|               |               | - tasty            |
+---------------+---------------+--------------------+
\end{verbatim}

The row of \texttt{=}s separates the header from the table body, and can
be omitted for a headerless table. The cells of grid tables may contain
arbitrary block elements (multiple paragraphs, code blocks, lists,
etc.). Alignments are not supported, nor are cells that span multiple
columns or rows. Grid tables can be created easily using
\href{http://table.sourceforge.net/}{Emacs table mode}.

\subsubsection{Pipe tables}

\textbf{Extension: \texttt{pipe\_tables}, \texttt{table\_captions}}

Pipe tables look like this:

\begin{verbatim}
| Right | Left | Default | Center |
|------:|:-----|---------|:------:|
|   12  |  12  |    12   |    12  |
|  123  |  123 |   123   |   123  |
|    1  |    1 |     1   |     1  |

  : Demonstration of simple table syntax.
\end{verbatim}

The syntax is
\href{http://michelf.ca/projects/php-markdown/extra/\#table}{the same as
in PHP markdown extra}. The beginning and ending pipe characters are
optional, but pipes are required between all columns. The colons
indicate column alignment as shown. The header can be omitted, but the
horizontal line must still be included, as it defines column alignments.

Since the pipes indicate column boundaries, columns need not be
vertically aligned, as they are in the above example. So, this is a
perfectly legal (though ugly) pipe table:

\begin{verbatim}
fruit| price
-----|-----:
apple|2.05
pear|1.37
orange|3.09
\end{verbatim}

The cells of pipe tables cannot contain block elements like paragraphs
and lists, and cannot span multiple lines.

Note: Pandoc also recognizes pipe tables of the following form, as can
produced by Emacs' orgtbl-mode:

\begin{verbatim}
| One | Two   |
|-----+-------|
| my  | table |
| is  | nice  |
\end{verbatim}

The difference is that \texttt{+} is used instead of
\texttt{\textbar{}}. Other orgtbl features are not supported. In
particular, to get non-default column alignment, you'll need to add
colons as above.

\subsection{Title block}

\textbf{Extension: \texttt{pandoc\_title\_block}}

If the file begins with a title block

\begin{verbatim}
% title
% author(s) (separated by semicolons)
% date
\end{verbatim}

it will be parsed as bibliographic information, not regular text. (It
will be used, for example, in the title of standalone LaTeX or HTML
output.) The block may contain just a title, a title and an author, or
all three elements. If you want to include an author but no title, or a
title and a date but no author, you need a blank line:

\begin{verbatim}
%
% Author

% My title
%
% June 15, 2006
\end{verbatim}

The title may occupy multiple lines, but continuation lines must begin
with leading space, thus:

\begin{verbatim}
% My title
  on multiple lines
\end{verbatim}

If a document has multiple authors, the authors may be put on separate
lines with leading space, or separated by semicolons, or both. So, all
of the following are equivalent:

\begin{verbatim}
% Author One
  Author Two

% Author One; Author Two

% Author One;
  Author Two
\end{verbatim}

The date must fit on one line.

All three metadata fields may contain standard inline formatting
(italics, links, footnotes, etc.).

Title blocks will always be parsed, but they will affect the output only
when the \texttt{-{}-standalone} (\texttt{-s}) option is chosen. In HTML
output, titles will appear twice: once in the document head -- this is
the title that will appear at the top of the window in a browser -- and
once at the beginning of the document body. The title in the document
head can have an optional prefix attached (\texttt{-{}-title-prefix} or
\texttt{-T} option). The title in the body appears as an H1 element with
class ``title'', so it can be suppressed or reformatted with CSS. If a
title prefix is specified with \texttt{-T} and no title block appears in
the document, the title prefix will be used by itself as the HTML title.

The man page writer extracts a title, man page section number, and other
header and footer information from the title line. The title is assumed
to be the first word on the title line, which may optionally end with a
(single-digit) section number in parentheses. (There should be no space
between the title and the parentheses.) Anything after this is assumed
to be additional footer and header text. A single pipe character
(\texttt{\textbar{}}) should be used to separate the footer text from
the header text. Thus,

\begin{verbatim}
% PANDOC(1)
\end{verbatim}

will yield a man page with the title \texttt{PANDOC} and section 1.

\begin{verbatim}
% PANDOC(1) Pandoc User Manuals
\end{verbatim}

will also have ``Pandoc User Manuals'' in the footer.

\begin{verbatim}
% PANDOC(1) Pandoc User Manuals | Version 4.0
\end{verbatim}

will also have ``Version 4.0'' in the header.

\subsection{Backslash escapes}

\textbf{Extension: \texttt{all\_symbols\_escapable}}

Except inside a code block or inline code, any punctuation or space
character preceded by a backslash will be treated literally, even if it
would normally indicate formatting. Thus, for example, if one writes

\begin{verbatim}
*\*hello\**
\end{verbatim}

one will get

\begin{verbatim}
<em>*hello*</em>
\end{verbatim}

instead of

\begin{verbatim}
<strong>hello</strong>
\end{verbatim}

This rule is easier to remember than standard markdown's rule, which
allows only the following characters to be backslash-escaped:

\begin{verbatim}
\`*_{}[]()>#+-.!
\end{verbatim}

(However, if the \texttt{markdown\_strict} format is used, the standard
markdown rule will be used.)

A backslash-escaped space is parsed as a nonbreaking space. It will
appear in TeX output as \texttt{\textasciitilde{}} and in HTML and XML
as \texttt{\textbackslash{}\&\#160;} or
\texttt{\textbackslash{}\&nbsp;}.

A backslash-escaped newline (i.e.~a backslash occurring at the end of a
line) is parsed as a hard line break. It will appear in TeX output as
\texttt{\textbackslash{}\textbackslash{}} and in HTML as
\texttt{\textless{}br /\textgreater{}}. This is a nice alternative to
markdown's ``invisible'' way of indicating hard line breaks using two
trailing spaces on a line.

Backslash escapes do not work in verbatim contexts.

\subsection{Smart punctuation}

\textbf{Extension}

If the \texttt{-{}-smart} option is specified, pandoc will produce
typographically correct output, converting straight quotes to curly
quotes, \texttt{-{}-{}-} to em-dashes, \texttt{-{}-} to en-dashes, and
\texttt{...} to ellipses. Nonbreaking spaces are inserted after certain
abbreviations, such as ``Mr.''

Note: if your LaTeX template uses the \texttt{csquotes} package, pandoc
will detect automatically this and use
\texttt{\textbackslash{}enquote\{...\}} for quoted text.

\hyperdef{}{inline-formatting}{\subsection{Inline
formatting}\label{inline-formatting}}

\subsubsection{Emphasis}

To \emph{emphasize} some text, surround it with \texttt{*}s or
\texttt{\_}, like this:

\begin{verbatim}
This text is _emphasized with underscores_, and this
is *emphasized with asterisks*.
\end{verbatim}

Double \texttt{*} or \texttt{\_} produces \textbf{strong emphasis}:

\begin{verbatim}
This is **strong emphasis** and __with underscores__.
\end{verbatim}

A \texttt{*} or \texttt{\_} character surrounded by spaces, or
backslash-escaped, will not trigger emphasis:

\begin{verbatim}
This is * not emphasized *, and \*neither is this\*.
\end{verbatim}

\textbf{Extension: \texttt{intraword\_underscores}}

Because \texttt{\_} is sometimes used inside words and identifiers,
pandoc does not interpret a \texttt{\_} surrounded by alphanumeric
characters as an emphasis marker. If you want to emphasize just part of
a word, use \texttt{*}:

\begin{verbatim}
feas*ible*, not feas*able*.
\end{verbatim}

\subsubsection{Strikeout}

\textbf{Extension: \texttt{strikeout}}

To strikeout a section of text with a horizontal line, begin and end it
with \texttt{\textasciitilde{}\textasciitilde{}}. Thus, for example,

\begin{verbatim}
This ~~is deleted text.~~
\end{verbatim}

\subsubsection{Superscripts and subscripts}

\textbf{Extension: \texttt{superscript}, \texttt{subscript}}

Superscripts may be written by surrounding the superscripted text by
\texttt{\^{}} characters; subscripts may be written by surrounding the
subscripted text by \texttt{\textasciitilde{}} characters. Thus, for
example,

\begin{verbatim}
H~2~O is a liquid.  2^10^ is 1024.
\end{verbatim}

If the superscripted or subscripted text contains spaces, these spaces
must be escaped with backslashes. (This is to prevent accidental
superscripting and subscripting through the ordinary use of
\texttt{\textasciitilde{}} and \texttt{\^{}}.) Thus, if you want the
letter P with `a cat' in subscripts, use
\texttt{P\textasciitilde{}a\textbackslash{} cat\textasciitilde{}}, not
\texttt{P\textasciitilde{}a cat\textasciitilde{}}.

\subsubsection{Verbatim}

To make a short span of text verbatim, put it inside backticks:

\begin{verbatim}
What is the difference between `>>=` and `>>`?
\end{verbatim}

If the verbatim text includes a backtick, use double backticks:

\begin{verbatim}
Here is a literal backtick `` ` ``.
\end{verbatim}

(The spaces after the opening backticks and before the closing backticks
will be ignored.)

The general rule is that a verbatim span starts with a string of
consecutive backticks (optionally followed by a space) and ends with a
string of the same number of backticks (optionally preceded by a space).

Note that backslash-escapes (and other markdown constructs) do not work
in verbatim contexts:

\begin{verbatim}
This is a backslash followed by an asterisk: `\*`.
\end{verbatim}

\textbf{Extension: \texttt{inline\_code\_attributes}}

Attributes can be attached to verbatim text, just as with
\hyperref[fenced-code-blocks]{fenced code blocks}:

\begin{verbatim}
`<$>`{.haskell}
\end{verbatim}

\subsection{Math}

\textbf{Extension: \texttt{tex\_math\_dollars}}

Anything between two \texttt{\$} characters will be treated as TeX math.
The opening \texttt{\$} must have a character immediately to its right,
while the closing \texttt{\$} must have a character immediately to its
left. Thus, \texttt{\$20,000 and \$30,000} won't parse as math. If for
some reason you need to enclose text in literal \texttt{\$} characters,
backslash-escape them and they won't be treated as math delimiters.

TeX math will be printed in all output formats. How it is rendered
depends on the output format:

\begin{description}
\item[Markdown, LaTeX, Org-Mode, ConTeXt]
It will appear verbatim between \texttt{\$} characters.
\item[reStructuredText]
It will be rendered using an interpreted text role \texttt{:math:}, as
described
\href{http://www.american.edu/econ/itex2mml/mathhack.rst}{here}.
\item[AsciiDoc]
It will be rendered as \texttt{latexmath:{[}...{]}}.
\item[Texinfo]
It will be rendered inside a \texttt{@math} command.
\item[groff man]
It will be rendered verbatim without \texttt{\$}'s.
\item[MediaWiki]
It will be rendered inside \texttt{\textless{}math\textgreater{}} tags.
\item[Textile]
It will be rendered inside
\texttt{\textless{}span class="math"\textgreater{}} tags.
\item[RTF, OpenDocument, ODT]
It will be rendered, if possible, using unicode characters, and will
otherwise appear verbatim.
\item[Docbook]
If the \texttt{-{}-mathml} flag is used, it will be rendered using
mathml in an \texttt{inlineequation} or \texttt{informalequation} tag.
Otherwise it will be rendered, if possible, using unicode characters.
\item[Docx]
It will be rendered using OMML math markup.
\item[FictionBook2]
If the \texttt{-{}-webtex} option is used, formulas are rendered as
images using Google Charts or other compatible web service, downloaded
and embedded in the e-book. Otherwise, they will appear verbatim.
\item[HTML, Slidy, DZSlides, S5, EPUB]
The way math is rendered in HTML will depend on the command-line options
selected:

\begin{enumerate}
\def\labelenumi{\arabic{enumi}.}
\item
  The default is to render TeX math as far as possible using unicode
  characters, as with RTF, DocBook, and OpenDocument output. Formulas
  are put inside a \texttt{span} with \texttt{class="math"}, so that
  they may be styled differently from the surrounding text if needed.
\item
  If the \texttt{-{}-latexmathml} option is used, TeX math will be
  displayed between \texttt{\$} or \texttt{\$\$} characters and put in
  \texttt{\textless{}span\textgreater{}} tags with class \texttt{LaTeX}.
  The \href{http://math.etsu.edu/LaTeXMathML/}{LaTeXMathML} script will
  be used to render it as formulas. (This trick does not work in all
  browsers, but it works in Firefox. In browsers that do not support
  LaTeXMathML, TeX math will appear verbatim between \texttt{\$}
  characters.)
\item
  If the \texttt{-{}-jsmath} option is used, TeX math will be put inside
  \texttt{\textless{}span\textgreater{}} tags (for inline math) or
  \texttt{\textless{}div\textgreater{}} tags (for display math) with
  class \texttt{math}. The
  \href{http://www.math.union.edu/~dpvc/jsmath/}{jsMath} script will be
  used to render it.
\item
  If the \texttt{-{}-mimetex} option is used, the
  \href{http://www.forkosh.com/mimetex.html}{mimeTeX} CGI script will be
  called to generate images for each TeX formula. This should work in
  all browsers. The \texttt{-{}-mimetex} option takes an optional URL as
  argument. If no URL is specified, it will be assumed that the mimeTeX
  CGI script is at \texttt{/cgi-bin/mimetex.cgi}.
\item
  If the \texttt{-{}-gladtex} option is used, TeX formulas will be
  enclosed in \texttt{\textless{}eq\textgreater{}} tags in the HTML
  output. The resulting \texttt{htex} file may then be processed by
  \href{http://ans.hsh.no/home/mgg/gladtex/}{gladTeX}, which will
  produce image files for each formula and an \texttt{html} file with
  links to these images. So, the procedure is:

\begin{verbatim}
pandoc -s --gladtex myfile.txt -o myfile.htex
gladtex -d myfile-images myfile.htex
# produces myfile.html and images in myfile-images
\end{verbatim}
\item
  If the \texttt{-{}-webtex} option is used, TeX formulas will be
  converted to \texttt{\textless{}img\textgreater{}} tags that link to
  an external script that converts formulas to images. The formula will
  be URL-encoded and concatenated with the URL provided. If no URL is
  specified, the Google Chart API will be used
  (\texttt{http://chart.apis.google.com/chart?cht=tx\&chl=}).
\item
  If the \texttt{-{}-mathjax} option is used, TeX math will be displayed
  between \texttt{\textbackslash{}(...\textbackslash{})} (for inline
  math) or \texttt{\textbackslash{}{[}...\textbackslash{}{]}} (for
  display math) and put in \texttt{\textless{}span\textgreater{}} tags
  with class \texttt{math}. The \href{http://www.mathjax.org/}{MathJax}
  script will be used to render it as formulas.
\end{enumerate}
\end{description}

\subsection{Raw HTML}

\textbf{Extension: \texttt{raw\_html}}

Markdown allows you to insert raw HTML (or DocBook) anywhere in a
document (except verbatim contexts, where \texttt{\textless{}},
\texttt{\textgreater{}}, and \texttt{\&} are interpreted literally).
(Techncially this is not an extension, since standard markdown allows
it, but it has been made an extension so that it can be disabled if
desired.)

The raw HTML is passed through unchanged in HTML, S5, Slidy, Slideous,
DZSlides, EPUB, Markdown, and Textile output, and suppressed in other
formats.

\textbf{Extension: \texttt{markdown\_in\_html\_blocks}}

Standard markdown allows you to include HTML ``blocks'': blocks of HTML
between balanced tags that are separated from the surrounding text with
blank lines, and start and end at the left margin. Within these blocks,
everything is interpreted as HTML, not markdown; so (for example),
\texttt{*} does not signify emphasis.

Pandoc behaves this way when the \texttt{markdown\_strict} format is
used; but by default, pandoc interprets material between HTML block tags
as markdown. Thus, for example, Pandoc will turn

\begin{verbatim}
<table>
    <tr>
        <td>*one*</td>
        <td>[a link](http://google.com)</td>
    </tr>
</table>
\end{verbatim}

into

\begin{verbatim}
<table>
    <tr>
        <td><em>one</em></td>
        <td><a href="http://google.com">a link</a></td>
    </tr>
</table>
\end{verbatim}

whereas \texttt{Markdown.pl} will preserve it as is.

There is one exception to this rule: text between
\texttt{\textless{}script\textgreater{}} and
\texttt{\textless{}style\textgreater{}} tags is not interpreted as
markdown.

This departure from standard markdown should make it easier to mix
markdown with HTML block elements. For example, one can surround a block
of markdown text with \texttt{\textless{}div\textgreater{}} tags without
preventing it from being interpreted as markdown.

\subsection{Raw TeX}

\textbf{Extension: \texttt{raw\_tex}}

In addition to raw HTML, pandoc allows raw LaTeX, TeX, and ConTeXt to be
included in a document. Inline TeX commands will be preserved and passed
unchanged to the LaTeX and ConTeXt writers. Thus, for example, you can
use LaTeX to include BibTeX citations:

\begin{verbatim}
This result was proved in \cite{jones.1967}.
\end{verbatim}

Note that in LaTeX environments, like

\begin{verbatim}
\begin{tabular}{|l|l|}\hline
Age & Frequency \\ \hline
18--25  & 15 \\
26--35  & 33 \\
36--45  & 22 \\ \hline
\end{tabular}
\end{verbatim}

the material between the begin and end tags will be interpreted as raw
LaTeX, not as markdown.

Inline LaTeX is ignored in output formats other than Markdown, LaTeX,
and ConTeXt.

\subsection{LaTeX macros}

\textbf{Extension: \texttt{latex\_macros}}

For output formats other than LaTeX, pandoc will parse LaTeX
\texttt{\textbackslash{}newcommand} and
\texttt{\textbackslash{}renewcommand} definitions and apply the
resulting macros to all LaTeX math. So, for example, the following will
work in all output formats, not just LaTeX:

\begin{verbatim}
\newcommand{\tuple}[1]{\langle #1 \rangle}

$\tuple{a, b, c}$
\end{verbatim}

In LaTeX output, the \texttt{\textbackslash{}newcommand} definition will
simply be passed unchanged to the output.

\subsection{Links}

Markdown allows links to be specified in several ways.

\subsubsection{Automatic links}

If you enclose a URL or email address in pointy brackets, it will become
a link:

\begin{verbatim}
<http://google.com>
<sam@green.eggs.ham>
\end{verbatim}

\subsubsection{Inline links}

An inline link consists of the link text in square brackets, followed by
the URL in parentheses. (Optionally, the URL can be followed by a link
title, in quotes.)

\begin{verbatim}
This is an [inline link](/url), and here's [one with
a title](http://fsf.org "click here for a good time!").
\end{verbatim}

There can be no space between the bracketed part and the parenthesized
part. The link text can contain formatting (such as emphasis), but the
title cannot.

\subsubsection{Reference links}

An \emph{explicit} reference link has two parts, the link itself and the
link definition, which may occur elsewhere in the document (either
before or after the link).

The link consists of link text in square brackets, followed by a label
in square brackets. (There can be space between the two.) The link
definition consists of the bracketed label, followed by a colon and a
space, followed by the URL, and optionally (after a space) a link title
either in quotes or in parentheses.

Here are some examples:

\begin{verbatim}
[my label 1]: /foo/bar.html  "My title, optional"
[my label 2]: /foo
[my label 3]: http://fsf.org (The free software foundation)
[my label 4]: /bar#special  'A title in single quotes'
\end{verbatim}

The URL may optionally be surrounded by angle brackets:

\begin{verbatim}
[my label 5]: <http://foo.bar.baz>
\end{verbatim}

The title may go on the next line:

\begin{verbatim}
[my label 3]: http://fsf.org
  "The free software foundation"
\end{verbatim}

Note that link labels are not case sensitive. So, this will work:

\begin{verbatim}
Here is [my link][FOO]

[Foo]: /bar/baz
\end{verbatim}

In an \emph{implicit} reference link, the second pair of brackets is
empty, or omitted entirely:

\begin{verbatim}
See [my website][], or [my website].

[my website]: http://foo.bar.baz
\end{verbatim}

Note: In \texttt{Markdown.pl} and most other markdown implementations,
reference link definitions cannot occur in nested constructions such as
list items or block quotes. Pandoc lifts this arbitrary seeming
restriction. So the following is fine in pandoc, though not in most
other implementations:

\begin{verbatim}
> My block [quote].
>
> [quote]: /foo
\end{verbatim}

\subsubsection{Internal links}

To link to another section of the same document, use the automatically
generated identifier (see
\hyperref[header-identifiers-in-html-latex-and-context]{Header
identifiers in HTML, LaTeX, and ConTeXt}, below). For example:

\begin{verbatim}
See the [Introduction](#introduction).
\end{verbatim}

or

\begin{verbatim}
See the [Introduction].

[Introduction]: #introduction
\end{verbatim}

Internal links are currently supported for HTML formats (including HTML
slide shows and EPUB), LaTeX, and ConTeXt.

\subsection{Images}

A link immediately preceded by a \texttt{!} will be treated as an image.
The link text will be used as the image's alt text:

\begin{verbatim}
![la lune](lalune.jpg "Voyage to the moon")

![movie reel]

[movie reel]: movie.gif
\end{verbatim}

\subsubsection{Pictures with captions}

\textbf{Extension: \texttt{implicit\_figures}}

An image occurring by itself in a paragraph will be rendered as a figure
with a caption.\footnote{This feature is not yet implemented for RTF,
  OpenDocument, or ODT. In those formats, you'll just get an image in a
  paragraph by itself, with no caption.} (In LaTeX, a figure environment
will be used; in HTML, the image will be placed in a \texttt{div} with
class \texttt{figure}, together with a caption in a \texttt{p} with
class \texttt{caption}.) The image's alt text will be used as the
caption.

\begin{verbatim}
![This is the caption](/url/of/image.png)
\end{verbatim}

If you just want a regular inline image, just make sure it is not the
only thing in the paragraph. One way to do this is to insert a
nonbreaking space after the image:

\begin{verbatim}
![This image won't be a figure](/url/of/image.png)\
\end{verbatim}

\subsection{Footnotes}

\textbf{Extension: \texttt{footnotes}}

Pandoc's markdown allows footnotes, using the following syntax:

\begin{verbatim}
Here is a footnote reference,[^1] and another.[^longnote]

[^1]: Here is the footnote.

[^longnote]: Here's one with multiple blocks.

    Subsequent paragraphs are indented to show that they
belong to the previous footnote.

        { some.code }

    The whole paragraph can be indented, or just the first
    line.  In this way, multi-paragraph footnotes work like
    multi-paragraph list items.

This paragraph won't be part of the note, because it
isn't indented.
\end{verbatim}

The identifiers in footnote references may not contain spaces, tabs, or
newlines. These identifiers are used only to correlate the footnote
reference with the note itself; in the output, footnotes will be
numbered sequentially.

The footnotes themselves need not be placed at the end of the document.
They may appear anywhere except inside other block elements (lists,
block quotes, tables, etc.).

\textbf{Extension: \texttt{inline\_notes}}

Inline footnotes are also allowed (though, unlike regular notes, they
cannot contain multiple paragraphs). The syntax is as follows:

\begin{verbatim}
Here is an inline note.^[Inlines notes are easier to write, since
you don't have to pick an identifier and move down to type the
note.]
\end{verbatim}

Inline and regular footnotes may be mixed freely.

\subsection{Citations}

\textbf{Extension: \texttt{citations}}

Pandoc can automatically generate citations and a bibliography in a
number of styles (using Andrea Rossato's \texttt{hs-citeproc}). In order
to use this feature, you will need a bibliographic database in one of
the following formats:

\begin{longtable}[c]{@{}ll@{}}
\hline\noalign{\medskip}
Format & File extension
\\\noalign{\medskip}
\hline\noalign{\medskip}
MODS & .mods
\\\noalign{\medskip}
BibLaTeX & .bib
\\\noalign{\medskip}
BibTeX & .bibtex
\\\noalign{\medskip}
RIS & .ris
\\\noalign{\medskip}
EndNote & .enl
\\\noalign{\medskip}
EndNote XML & .xml
\\\noalign{\medskip}
ISI & .wos
\\\noalign{\medskip}
MEDLINE & .medline
\\\noalign{\medskip}
Copac & .copac
\\\noalign{\medskip}
JSON citeproc & .json
\\\noalign{\medskip}
\hline
\end{longtable}

Note that \texttt{.bib} can generally be used with both BibTeX and
BibLaTeX files, but you can use \texttt{.bibtex} to force BibTeX.

You will need to specify the bibliography file using the
\texttt{-{}-bibliography} command-line option (which may be repeated if
you have several bibliographies).

By default, pandoc will use a Chicago author-date format for citations
and references. To use another style, you will need to use the
\texttt{-{}-csl} option to specify a
\href{http://CitationStyles.org}{CSL} 1.0 style file. A primer on
creating and modifying CSL styles can be found at
\url{http://citationstyles.org/downloads/primer.html}. A repository of
CSL styles can be found at
\url{https://github.com/citation-style-language/styles}. See also
\url{http://zotero.org/styles} for easy browsing.

Citations go inside square brackets and are separated by semicolons.
Each citation must have a key, composed of `@' + the citation identifier
from the database, and may optionally have a prefix, a locator, and a
suffix. Here are some examples:

\begin{verbatim}
Blah blah [see @doe99, pp. 33-35; also @smith04, ch. 1].

Blah blah [@doe99, pp. 33-35, 38-39 and *passim*].

Blah blah [@smith04; @doe99].
\end{verbatim}

A minus sign (\texttt{-}) before the \texttt{@} will suppress mention of
the author in the citation. This can be useful when the author is
already mentioned in the text:

\begin{verbatim}
Smith says blah [-@smith04].
\end{verbatim}

You can also write an in-text citation, as follows:

\begin{verbatim}
@smith04 says blah.

@smith04 [p. 33] says blah.
\end{verbatim}

If the style calls for a list of works cited, it will be placed at the
end of the document. Normally, you will want to end your document with
an appropriate header:

\begin{verbatim}
last paragraph...

# References
\end{verbatim}

The bibliography will be inserted after this header.

\section{Non-pandoc extensions}

The following markdown syntax extensions are not enabled by default in
pandoc, but may be enabled by adding \texttt{+EXTENSION} to the format
name, where \texttt{EXTENSION} is the name of the extension. Thus, for
example, \texttt{markdown+hard\_line\_breaks} is markdown with hard line
breaks.

\textbf{Extension: \texttt{hard\_line\_breaks}}\\Causes all newlines
within a paragraph to be interpreted as hard line breaks instead of
spaces.

\textbf{Extension: \texttt{tex\_math\_single\_backslash}}\\Causes
anything between \texttt{\textbackslash{}(} and
\texttt{\textbackslash{})} to be interpreted as inline TeX math, and
anything between \texttt{\textbackslash{}{[}} and
\texttt{\textbackslash{}{]}} to be interpreted as display TeX math.
Note: a drawback of this extension is that it precludes escaping
\texttt{(} and \texttt{{[}}.

\textbf{Extension: \texttt{tex\_math\_double\_backslash}}\\Causes
anything between \texttt{\textbackslash{}\textbackslash{}(} and
\texttt{\textbackslash{}\textbackslash{})} to be interpreted as inline
TeX math, and anything between
\texttt{\textbackslash{}\textbackslash{}{[}} and
\texttt{\textbackslash{}\textbackslash{}{]}} to be interpreted as
display TeX math.

\textbf{Extension: \texttt{markdown\_attribute}}\\By default, pandoc
interprets material inside block-level tags as markdown. This extension
changes the behavior so that markdown is only parsed inside block-level
tags if the tags have the attribute \texttt{markdown=1}.

\textbf{Extension: \texttt{mmd\_title\_block}}\\Enables a
\href{http://fletcherpenney.net/multimarkdown/}{MultiMarkdown} style
title block at the top of the document, for example:

\begin{verbatim}
Title:   My title
Author:  John Doe
Date:    September 1, 2008
Comment: This is a sample mmd title block, with
         a field spanning multiple lines.
\end{verbatim}

See the MultiMarkdown documentation for details. Note that only title,
author, and date are recognized; other fields are simply ignored by
pandoc. If \texttt{pandoc\_title\_block} is enabled, it will take
precedence over \texttt{mmd\_title\_block}.

\textbf{Extension: \texttt{abbrevations}}\\Parses PHP Markdown Extra
abbreviation keys, like

\begin{verbatim}
*[HTML]: Hyper Text Markup Language
\end{verbatim}

Note that the pandoc document model does not support abbreviations, so
if this extension is enabled, abbreviation keys are simply skipped (as
opposed to being parsed as paragraphs).

\textbf{Extension: \texttt{autolink\_bare\_uris}}\\Makes all absolute
URIs into links, even when not surrounded by pointy braces
\texttt{\textless{}...\textgreater{}}.

\textbf{Extension: \texttt{link\_attributes}}\\Parses multimarkdown
style key-value attributes on link and image references. Note that
pandoc's internal document model provides nowhere to put these, so they
are presently just ignored.

\textbf{Extension: \texttt{mmd\_header\_identifiers}}\\Parses
multimarkdown style header identifiers (in square brackets, after the
header but before any trailing \texttt{\#}s in an ATX header).

\section{Producing slide shows with Pandoc}

You can use Pandoc to produce an HTML + javascript slide presentation
that can be viewed via a web browser. There are four ways to do this,
using \href{http://meyerweb.com/eric/tools/s5/}{S5},
\href{http://paulrouget.com/dzslides/}{DZSlides},
\href{http://www.w3.org/Talks/Tools/Slidy/}{Slidy}, or
\href{http://goessner.net/articles/slideous/}{Slideous}. You can also
produce a PDF slide show using LaTeX
\href{http://www.tex.ac.uk/CTAN/macros/latex/contrib/beamer}{beamer}.

Here's the markdown source for a simple slide show, \texttt{habits.txt}:

\begin{verbatim}
% Habits
% John Doe
% March 22, 2005

# In the morning

## Getting up

- Turn off alarm
- Get out of bed

## Breakfast

- Eat eggs
- Drink coffee

# In the evening

## Dinner

- Eat spaghetti
- Drink wine

------------------

![picture of spaghetti](images/spaghetti.jpg)

## Going to sleep

- Get in bed
- Count sheep
\end{verbatim}

To produce the slide show, simply type

\begin{verbatim}
pandoc -t s5 -s habits.txt -o habits.html
\end{verbatim}

for S5,

\begin{verbatim}
pandoc -t slidy -s habits.txt -o habits.html
\end{verbatim}

for Slidy,

\begin{verbatim}
pandoc -t slideous -s habits.txt -o habits.html
\end{verbatim}

for Slideous,

\begin{verbatim}
pandoc -t dzslides -s habits.txt -o habits.html
\end{verbatim}

for DZSlides, or

\begin{verbatim}
pandoc -t beamer habits.txt -o habits.pdf
\end{verbatim}

for beamer.

With all HTML slide formats, the \texttt{-{}-self-contained} option can
be used to produce a single file that contains all of the data necessary
to display the slide show, including linked scripts, stylesheets,
images, and videos.

\hyperdef{}{structuring-the-slide-show}{\subsection{Structuring the
slide show}\label{structuring-the-slide-show}}

By default, the \emph{slide level} is the highest header level in the
hierarchy that is followed immediately by content, and not another
header, somewhere in the document. In the example above, level 1 headers
are always followed by level 2 headers, which are followed by content,
so 2 is the slide level. This default can be overridden using the
\texttt{-{}-slide-level} option.

The document is carved up into slides according to the following rules:

\begin{itemize}
\item
  A horizontal rule always starts a new slide.
\item
  A header at the slide level always starts a new slide.
\item
  Headers \emph{below} the slide level in the hierarchy create headers
  \emph{within} a slide.
\item
  Headers \emph{above} the slide level in the hierarchy create ``title
  slides,'' which just contain the section title and help to break the
  slide show into sections.
\item
  A title page is constructed automatically from the document's title
  block, if present. (In the case of beamer, this can be disabled by
  commenting out some lines in the default template.)
\end{itemize}

These rules are designed to support many different styles of slide show.
If you don't care about structuring your slides into sections and
subsections, you can just use level 1 headers for all each slide. (In
that case, level 1 will be the slide level.) But you can also structure
the slide show into sections, as in the example above.

For Slidy, Slideous and S5, the file produced by pandoc with the
\texttt{-s/-{}-standalone} option embeds a link to javascripts and CSS
files, which are assumed to be available at the relative path
\texttt{s5/default} (for S5) or \texttt{slideous} (for Slideous), or at
the Slidy website at \texttt{w3.org} (for Slidy). (These paths can be
changed by setting the \texttt{slidy-url}, \texttt{slideous-url} or
\texttt{s5-url} variables; see \texttt{-{}-variable}, above.) For
DZSlides, the (relatively short) javascript and css are included in the
file by default.

\subsection{Incremental lists}

By default, these writers produces lists that display ``all at once.''
If you want your lists to display incrementally (one item at a time),
use the \texttt{-i} option. If you want a particular list to depart from
the default (that is, to display incrementally without the \texttt{-i}
option and all at once with the \texttt{-i} option), put it in a block
quote:

\begin{verbatim}
> - Eat spaghetti
> - Drink wine
\end{verbatim}

In this way incremental and nonincremental lists can be mixed in a
single document.

\subsection{Styling the slides}

You can change the style of HTML slides by putting customized CSS files
in \texttt{\$DATADIR/s5/default} (for S5), \texttt{\$DATADIR/slidy} (for
Slidy), or \texttt{\$DATADIR/slideous} (for Slideous), where
\texttt{\$DATADIR} is the user data directory (see
\texttt{-{}-data-dir}, above). The originals may be found in pandoc's
system data directory (generally
\texttt{\$CABALDIR/pandoc-VERSION/s5/default}). Pandoc will look there
for any files it does not find in the user data directory.

For dzslides, the CSS is included in the HTML file itself, and may be
modified there.

To style beamer slides, you can specify a beamer ``theme'' or
``colortheme'' using the \texttt{-V} option:

\begin{verbatim}
pandoc -t beamer habits.txt -V theme:Warsaw -o habits.pdf
\end{verbatim}

\hyperdef{}{literate-haskell-support}{\section{Literate Haskell
support}\label{literate-haskell-support}}

If you append \texttt{+lhs} (or \texttt{+literate\_haskell}) to an
appropriate input or output format (\texttt{markdown},
\texttt{mardkown\_strict}, \texttt{rst}, or \texttt{latex} for input or
output; \texttt{beamer}, \texttt{html} or \texttt{html5} for output
only), pandoc will treat the document as literate Haskell source. This
means that

\begin{itemize}
\item
  In markdown input, ``bird track'' sections will be parsed as Haskell
  code rather than block quotations. Text between
  \texttt{\textbackslash{}begin\{code\}} and
  \texttt{\textbackslash{}end\{code\}} will also be treated as Haskell
  code.
\item
  In markdown output, code blocks with classes \texttt{haskell} and
  \texttt{literate} will be rendered using bird tracks, and block
  quotations will be indented one space, so they will not be treated as
  Haskell code. In addition, headers will be rendered setext-style (with
  underlines) rather than atx-style (with `\#' characters). (This is
  because ghc treats `\#' characters in column 1 as introducing line
  numbers.)
\item
  In restructured text input, ``bird track'' sections will be parsed as
  Haskell code.
\item
  In restructured text output, code blocks with class \texttt{haskell}
  will be rendered using bird tracks.
\item
  In LaTeX input, text in \texttt{code} environments will be parsed as
  Haskell code.
\item
  In LaTeX output, code blocks with class \texttt{haskell} will be
  rendered inside \texttt{code} environments.
\item
  In HTML output, code blocks with class \texttt{haskell} will be
  rendered with class \texttt{literatehaskell} and bird tracks.
\end{itemize}

Examples:

\begin{verbatim}
pandoc -f markdown+lhs -t html
\end{verbatim}

reads literate Haskell source formatted with markdown conventions and
writes ordinary HTML (without bird tracks).

\begin{verbatim}
pandoc -f markdown+lhs -t html+lhs
\end{verbatim}

writes HTML with the Haskell code in bird tracks, so it can be copied
and pasted as literate Haskell source.

\section{Authors}

© 2006-2013 John MacFarlane (jgm at berkeley dot edu). Released under
the \href{http://www.gnu.org/copyleft/gpl.html}{GPL}, version 2 or
greater. This software carries no warranty of any kind. (See COPYRIGHT
for full copyright and warranty notices.) Other contributors include
Recai Oktaş, Paulo Tanimoto, Peter Wang, Andrea Rossato, Eric Kow,
infinity0x, Luke Plant, shreevatsa.public, Puneeth Chaganti, Paul
Rivier, rodja.trappe, Bradley Kuhn, thsutton, Nathan Gass, Jonathan
Daugherty, Jérémy Bobbio, Justin Bogner, qerub, Christopher Sawicki,
Kelsey Hightower, Masayoshi Takahashi, Antoine Latter, Ralf Stephan,
Eric Seidel, B. Scott Michel, Gavin Beatty, Sergey Astanin, Arlo
O'Keeffe, Denis Laxalde, Brent Yorgey.

\end{document}
